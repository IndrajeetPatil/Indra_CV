\documentclass[10pt]{article}
	\usepackage{amssymb}
	\usepackage{hyperref}
	\usepackage{color}
	\usepackage{fontawesome}
	\usepackage{longtable}
	\usepackage[document]{ragged2e}
	
	\DeclareFontFamily{U}{fontawesomeOne}{}
	\DeclareFontShape{U}{fontawesomeOne}{m}{n}
  		{<-> FontAwesome--fontawesomeone}{}
	\DeclareRobustCommand\FAone{\fontencoding{U}\fontfamily{fontawesomeOne}\selectfont}
	
	%\setlength{\headheight}{0.0in}
	\setlength{\headheight}{-0.4in}
	\setlength{\headsep}{-0.3in}
	\addtolength{\topmargin}{0.1in}
	%\setlength{\topskip}{0.8in}
	%\setlength{\topskip}{0.0in}
	%\setlength{\topskip}{-1.0in}
	%\setlength{\topskip}{-0.7in}
	%\setlength{\footskip}{-0.6in}
	%\setlength{\textheight}{10.0in}
	\setlength{\textheight}{9.0in}
	\setlength{\oddsidemargin}{0.0in}
	\setlength{\evensidemargin}{0.0in}
	\setlength{\textwidth}{6.8in}
	
	\clubpenalty 10000
	\widowpenalty 10000
	\sloppy
	
	\newcommand{\comment}[1]{}
	\renewcommand{\baselinestretch}{1.0}
	
	\begin{document}
	
	\raggedright
	
	\setlength{\parindent}{0.0in}
	
	\setlength{\leftmargini}{0.25in}
	\setlength{\leftmarginii}{0.25in}
	
	\newcommand{\heading}[1]{
	 \raggedright
	\begin{flushleft}
	 \vspace{0.02in}
	 {\sffamily\large\textsc{#1}}\\
	 \rule[0.115in]{6.8in}{0.01in}
	 \vspace{-0.365in}
	\end{flushleft}
	}
	\newcommand{\elskip}
	{\vspace*{0.03in}}
	
	\newcommand{\mins}
	{\vspace*{-0.1in}}
	
	%\newcommand{}
	{\vspace*{0.07in}}
	\newcommand{\header}{%
	\raggedright
	\begin{flushleft}
	\parbox[c]{1.95in}{
	 \raggedright
	\begin{flushleft}
	Postdoctoral Fellow,\\
	Max Planck Institute\\
	for Human Development,\\
	Berlin, Germany.
	\end{flushleft}
	}
	\hfill
	\begin{minipage}[c]{2.25in}	
	 \begin{center}
	  { \Large \textsc{\textcolor{blue}{Indrajeet Patil}}\\
	  \begin{small}
	  (\tt patilindrajeet.science@gmail.com)\\
	\texttt{\today}
	\end{small}}
	 \end{center}
	\end{minipage}
	\hfill
	\parbox[c]{1.8in}
	{
	\begin{flushright}
	\raggedleft
	 \href{https://goo.gl/ss7v6C}{\Huge \faGithub} ~~ \href{https://www.linkedin.com/in/indrajeet-patil-397865174/}{\Huge \faLinkedinSquare} ~~ \href{https://twitter.com/patilindrajeets}{\Huge \faTwitterSquare} ~~ \href{https://stackoverflow.com/users/7973626/indrajeet-patil}{\Huge \faStackOverflow} 
	\end{flushright}
	}
	\end{flushleft}
	\vspace*{-0.2in}
	}%
	\newcommand{\smallheader}{
	 \begin{center}
	  {
	 \Large \textsc{Indrajeet S. Patil}}
	  \end{center}
	}
	\thispagestyle{empty}
	
	\header
	
	\heading{\textbf{Employment}}
    
    \begin{itemize}
	
	\item \textbf{\textcolor{red}{Postdoctoral Fellow}}, \textcolor{blue}{Max-Planck Institute for Human Development}, Berlin, Germany, 2019-.\\
	Advisors: Iyad Rahwan
	
	
	\item \textbf{\textcolor{red}{Postdoctoral Fellow}}, Department of Psychology, \textcolor{blue}{Harvard University}, Cambridge, USA, 2017-2019.\\
	Advisors: Mina Cikara and Fiery Cushman
	
	
	\item \textbf{\textcolor{red}{Visiting Researcher}}, Scuola Internazionale Superiore di Studi Avanzati (\textcolor{blue}{SISSA}), Trieste, Italy, 2016.
	%Advisor: Giorgia Silani
	
	\end{itemize}
	
	\heading{\textbf{Education}}
	
	\begin{itemize}
	
	\item \textbf{\textcolor{red}{Doctor of Philosophy}} (Neuroscience), \textcolor{blue}{SISSA}, Trieste, Italy, 2011-2015 (\textit{summa cum laude}).\\ 
	Thesis: \href{https://drive.google.com/open?id=1g8L7pf-SutYTWt-8yJ44bV96cixMBZuh}{\it A multimodal investigation of moral decision making in harmful contexts} (Advisor: Giorgia Silani).
	 
	
	\item \textbf{\textcolor{red}{Master of Science}} (Physics), Department of Physics, \textcolor{blue}{University of Pune}, India, 2008-2010 (GPA: 4.2/6).\\
	Thesis: \href{https://drive.google.com/open?id=0B6_u70YpdJKnMTJEMkI2RkQyNjc2QjJFOTowLjEx}{\it Effect of hypergravity on amylase and total protein content in germinating rice seeds}.
	
	
	\item \textbf{\textcolor{red}{Bachelor of Science}} (Physics), \textcolor{blue}{Fergusson College}, India, 2005-2008 (Score: 83$\%$).\\
	Thesis: \href{https://drive.google.com/open?id=0B6_u70YpdJKnMTIxNTkyODMxMzBBMkYyMDowLjEz}{\it Theoretical studies of static friction at atomic scale for unlubricated BCC solids}.\\
	{\it Satish Bhide Prize} for the Best Undergraduate Thesis 
	
	%\item Higher Secondary Certificate Exam, Maharashtra State Board, 2005, Science (Score: 95.33$\%$).
	
	
	%School Leaving Examination, Maharashtra State Board, 2003. (Score: 87.73$\%$)\\
	%{\it Ranked $4^{th}$ in the school}
	% 
    \end{itemize}
    
	\heading{\textbf{Languages}}
	\textbf{Marathi} (C2), \textbf{Hindi} (C2), \textbf{English} (C2), \textbf{German} (B2), \textbf{Italian} (A2)
	
\heading{\textbf{Publications}} 

	\begin{center}
	(*equal contribution, \href{https://scholar.google.it/citations?user=kSYuYTUAAAAJ&hl=en&oi=ao}{Citation count}: \textbf{\textcolor{red}{977}}, \textit{h}-index: \textbf{\textcolor{red}{14}}, \textit{i}10-index: \textbf{\textcolor{red}{17}}, \href{https://osf.io/hk5f3/}{Open data/scripts}: \textbf{\textcolor{red}{13/13}} publications)
	\end{center}
	
	\begin{itemize}
	
	\item \textbf{\textcolor{red}{Patil, I.}}*, Zucchelli, M. M.*, Kool, W., Campbell, S., Fornasier, F., Cal\`{o}, M., Silani, G., Cikara, M., \& Cushman, F. A. (2021). Reasoning supports utilitarian resolutions to moral dilemmas across diverse measures. \textit{\textcolor{blue}{Journal of Personality and Social Psychology}}, \textit{120}(2), 443–460.. \href{https://www.researchgate.net/publication/338496843_Reasoning_supports_utilitarian_resolutions_to_moral_dilemmas_across_diverse_measures}{\tt doi:10.1037/pspp0000281}
	
	\item Dhaliwal, N. A.*, \textbf{\textcolor{red}{Patil, I.}}*, \& Cushman, F. A. (2021). Reputational and cooperative benefits of third-party compensation. \textit{\textcolor{blue}{Organizational Behavior and Human Decision Processes}}, \textit{164}, 27-51. \href{https://www.researchgate.net/publication/349073655_Reputational_and_cooperative_benefits_of_third-party_compensation}{\tt doi:10.17605/OSF.IO/C3BSJ}
	
	\item  \textbf{\textcolor{red}{Patil, I.}}*, Larsen, E.*, Kichic, R., \& Gleichgerrcht, E. (2020). Moral cognition about harm in anxiety  disorders. \textit{\textcolor{blue}{Psychological Reports}}. \href{https://psyarxiv.com/g5p7v/}{\tt doi:10.1177/0033294120964134}
	
    \item Lüdecke, D., Ben-Shachar,  M. S., \textbf{\textcolor{red}{Patil, I.}}, \& Makowski, D. (2020). Extracting, Computing and Exploring the Parameters of Statistical Models using R. \textit{\textcolor{blue}{Journal of Open Source Software}}, \textit{5}(51), 2445. \href{https://joss.theoj.org/papers/10.21105/joss.02445}{\tt https://doi.org/10.21105/joss.02445}
	
    \item Makowski, D., Ben-Shachar,  M. S., \textbf{\textcolor{red}{Patil, I.}}, \& Lüdecke, D. (2020). Methods and Algorithms for Correlation Analysis in R. \textit{\textcolor{blue}{Journal of Open Source Software}}, \textit{5}(51), 2306. \href{https://joss.theoj.org/papers/10.21105/joss.02306}{\tt https://doi.org/10.21105/joss.02306}
	
	\item \textbf{\textcolor{red}{Patil, I.}}, Zanon, M.*, Novembre, G.*, Zangrando, N., Chittaro, L., \& Silani, G. (2018). Neuroanatomical basis of concern-based altruism in virtual environment. \textit{\textcolor{blue}{Neuropsychologia}}, \textit{116}, 34-43. \href{https://drive.google.com/open?id=0B6_u70YpdJKnWU0zblpBOUMxRXc}{\tt doi:10.1016/j.neuropsychologia.2017.02.015}
		
	\item \textbf{\textcolor{red}{Patil, I.}}, Cal\`{o}, M., Fornasier, F., Cushman, F. A.*, \& Silani, G.* (2017). The behavioral and neural basis of empathic blame. 
 \textit{\textcolor{blue}{Scientific Reports}}, \textit{7}:5200. \href{https://drive.google.com/open?id=0B6_u70YpdJKnT1J6dkk5R1NPZDg}{\tt doi:10.1038/s41598-017-05299-9}
	
	\item \textbf{\textcolor{red}{Patil, I.}}, Cal\`{o}, M.*, Fornasier, F.*, Young, L., \& Silani, G. (2017). Neuroanatomical correlates of forgiving unintentional harms. \textit{\textcolor{blue}{Scientific Reports}}, \textit{7}:45967. \href{https://drive.google.com/open?id=0B6_u70YpdJKnRV81UnNlZHdrdXM}{\tt doi:10.1038/srep45967}
	
	\item \textbf{\textcolor{red}{Patil, I.}}, Young, L., Sinay, V., \& Gleichgerrcht, E. (2017). Elevated moral condemnation of third-party violations in multiple sclerosis patients. \textit{\textcolor{blue}{Social Neuroscience}}, \textit{12}(3), 308-329. \href{https://drive.google.com/open?id=0B6_u70YpdJKnT3M0bWpPdDY4RHc}{\tt doi:10.1080/17470919.2016.1175380}
	
	\item \textbf{\textcolor{red}{Patil, I.}}*, Melsbach, J.*, Hennig-Fast, K., \& Silani, G. (2016). Divergent roles of autistic and alexithymic traits in utilitarian moral judgments in adults with autism. \textit{\textcolor{blue}{Scientific Reports}}, \textit{6}:23637.\\
	\href{https://drive.google.com/file/d/0B6_u70YpdJKnV1p2UmNwa09iS1k/view?usp=sharing}{\tt doi:10.1038/srep23637}
	
	\item \textbf{\textcolor{red}{Patil, I.}} (2015). Trait psychopathy and utilitarian moral judgement: The mediating role of action aversion. \textit{\textcolor{blue}{Journal of Cognitive Psychology}}, \textit{27}(3), 349-366. \href{https://drive.google.com/file/d/0B6_u70YpdJKnQjVJUFgtR2ZmWTVqZGxRMEFfemVIOUVuME5Z/view?usp=sharing}{\tt doi:10.1080/20445911.2015.1004334}
	
	\item \textbf{\textcolor{red}{Patil, I.}}, \& Silani, G. (2014). Alexithymia increases moral acceptability of accidental harms. \textit{\textcolor{blue}{Journal of Cognitive Psychology}}, \textit{26}(5), 597-614. \href{https://drive.google.com/file/d/0B6_u70YpdJKnMU5pVHRUM3p0SFk/view?usp=sharing}{\tt doi:10.1080/20445911.2014.929137}
	
	\item \textbf{\textcolor{red}{Patil, I.}}, \& Silani, G. (2014). Reduced empathic concern leads to utilitarian moral judgments in trait alexithymia. \textit{\textcolor{blue}{Frontiers in Psychology}}, \textit{5}:501. \href{https://drive.google.com/file/d/0B6_u70YpdJKnUkJZR252dXBwcVk/view?usp=sharing}{\tt doi:10.3389/fpsyg.2014.00501}
	
	\item \textbf{\textcolor{red}{Patil, I.}}, Cogoni, C., Zangrando, N., Chittaro, L., \& Silani, G. (2014). {Affective basis of judgment-behavior discrepancy in virtual experiences of moral dilemmas}, \textit{\textcolor{blue}{Social Neuroscience}}, \textit{9}(1), 94-107. \\
	\href{https://drive.google.com/file/d/0B6_u70YpdJKnV3RIR2s2cWlFdVU/view?usp=sharing}{\tt doi:10.1080/17470919.2013.870091}
	
	\end{itemize}
	
%	\heading{\textbf{Preprints}} 
	
%	\item \textbf{\textcolor{red}{Patil, I.}}, \& Tr\'{e}moli\`{e}re, B. (Under review). I think, therefore I forgive: The role of reasoning in forgiving accidental harms. \textit{\textcolor{blue}{Scientific Reports}}. \href{https://psyarxiv.com/8gjth/}{\tt doi:10.17605/OSF.IO/8GJTH}
	

%	\item Melsbach, J.*, \textbf{\textcolor{red}{Patil, I.}}*, Hennig-Fast, K., \& Silani, G. (in preparation). %Preserved intent-based moral judgments in high-functioning adults with autism.
	
%	\heading{\textbf{Selected Media Coverage}}
%	\item \href{http://time.com/3242/driving-over-your-best-friend-its-the-right-thing-to-do/}{Driving Over Your Best Friend: It's the Right Thing to Do, \textcolor{blue}{\textit{Time}}, January 15, 2014}.
%	
%	
%	\item \href{http://www.huffingtonpost.com/entry/autism-empathy-brain-research_us_56f92575e4b014d3fe237413}{We May Have Been Wrong About Autism And Empathy, \textcolor{blue}{\textit{The Huffington Post}}, March 29, 2016}.
%	
%	
%	\item \href{http://www.dailymail.co.uk/sciencetech/article-4308284/Virtual-reality-experiment-puts-altruism-test.html}{The brain of a HERO: Virtual reality test reveals what makes some people risk their lives to save others, \textcolor{blue}{\textit{Daily Mail}}, March 13, 2017}.
%	
%	
%	\item \href{http://www.repubblica.it/scienze/2017/04/11/news/area_cervello_perdono-162669836/?rss}{Il perdono \`{e} scritto nella mente: scoperta l'area del cervello, \textcolor{blue}{\textit{la Repubblica}}, April 11, 2017}.
	
	\heading{\textbf{Selected Media Coverage}}
	\href{http://time.com/3242/driving-over-your-best-friend-its-the-right-thing-to-do/}{\textit{Time}}, \href{http://www.huffingtonpost.com/entry/autism-empathy-brain-research_us_56f92575e4b014d3fe237413}{\textit{The Huffington Post}}, \href{http://www.dailymail.co.uk/sciencetech/article-4308284/Virtual-reality-experiment-puts-altruism-test.html}{\textit{Daily Mail}}, \href{http://www.repubblica.it/scienze/2017/04/11/news/area_cervello_perdono-162669836/?rss}{\textit{la Repubblica}}, etc.
	
	\heading{\textbf{Technical Skills}}
	
	\begin{itemize}
	\item \textcolor{red}{\textbf{Programming skills}}\\
	 Data analysis: \textcolor{blue}{R} (advanced), \textcolor{blue}{MATLAB} (intermediate), \textcolor{blue}{Python} (beginner)\\
	 
	 Database management: \textcolor{blue}{SQL} (intermediate)\\
	 
	 Web development: \textcolor{blue}{HTML/CSS} (intermediate), \textcolor{blue}{JavaScript} (intermediate)\\
		Document preparation: \textcolor{blue}{RMarkdown} (advanced), \textcolor{blue}{xaringan} (advanced), \textcolor{blue}{\LaTeX{}} (intermediate)\\
		
	 Miscellaneous: \textcolor{blue}{Git/GitHub}, \textcolor{blue}{YAML} 
	

	\item \textcolor{blue}{\textbf{Techniques}}: text mining, meta-analysis, logistic/linear regression, mixed effects regression, clustering (\textit{k}-means, hierarchical cluster analysis, linear discriminant analysis), \textit{k}-nearest neighbor, bagging, random forests, SVM, dimensionality reduction (PCA, ICA), Reinforcement Learning, Bayesian statistics, cross validation, neural networks, Structural Equation Modeling, path analysis
	
	
	\item \textcolor{blue}{\textbf{Open-source software development}} (in R; Total downloads: \textcolor{blue}{\textbf{$>$ 7M}}):
	 
	 \textbf{Lead developer}
	 
	\begin{itemize}
	
	 \vspace*{-0.09in}
	\itemsep-0.1em
	\item[--] \href{https://indrajeetpatil.github.io/ggstatsplot/}{\textcolor{red}{ggstatsplot}} (2018-): Informative data visualizations with statistical analyses.
%	\item[--] \href{https://indrajeetpatil.github.io/groupedstats/}{\textcolor{red}{groupedstats}} (2018-): Grouped statistical analyses.
	\item[--] \href{https://indrajeetpatil.github.io/broomExtra/}{\textcolor{red}{broomExtra}} (2019-): Enhancements for working with regression models.
	\item[--] \href{https://indrajeetpatil.github.io/statsExpressions/}{\textcolor{red}{statsExpressions}} (2019-): Creates expressions with statistical details.
	\item[--] \href{https://indrajeetpatil.github.io/pairwiseComparisons/}{\textcolor{red}{pairwiseComparisons}} (2019-): Multiple pairwise comparison tests.
	
	\end{itemize}
	
	 \textbf{Team developer}
	 \vspace*{-0.09in}
	
	\begin{itemize}	 
		\item[--] Co-author on \href{https://easystats.github.io/correlation/}{\textcolor{red}{correlation}}, \href{https://easystats.github.io/insight/}{\textcolor{red}{insight}}, \href{https://easystats.github.io/parameters/}{\textcolor{red}{parameters}}, \href{https://easystats.github.io/see/}{\textcolor{red}{see}}, \href{https://easystats.github.io/report/}{\textcolor{red}{report}}, \href{https://easystats.github.io/performance/}{\textcolor{red}{performance}}, \href{https://easystats.github.io/bayestestR/}{\textcolor{red}{bayestestR}},  \href{https://const-ae.github.io/ggsignif/}{\textcolor{red}{ggsignif}}

	\end{itemize}

	\item \textcolor{red}{\textbf{Statistical Analysis softwares}}\\
	\textit{Classical}: \textcolor{blue}{SPSS}, \textcolor{blue}{jamovi}; \textit{Bayesian}: \textcolor{blue}{JASP}; \textit{Structural Equation Modeling}: \textcolor{blue}{Amos}; \textit{Power analysis}: \textcolor{blue}{G*Power}
	
	
	\item \textcolor{red}{\textbf{Functional/Structural Magnetic Resonance Imaging}}: \textcolor{blue}{SPM12}\\
	\textit{Connectivity analysis}: PPI (task-based), GIFT (model-free); \textit{Morphometry analyses}: CAT12; \textit{Power analysis}: fmripower; \textit{Artifact handling}: ART, ArtRepair; \textit{Permutation tests}: SnPM13, TFCE; \textit{ROI analysis}: MarsBar;\\ 
	\textit{Data visualization}: Mango, bspmview
	
	
	\item \textcolor{red}{\textbf{Online data collection}}: \textcolor{blue}{Qualtrics}, \textcolor{blue}{MTurk}, \textcolor{blue}{TurkPrime}
	
	
	\item \textcolor{red}{\textbf{Skin conductance data analysis}}: \textcolor{blue}{Ledalab}

    \end{itemize}

	\heading{\textbf{Honors and Awards}}
	
	\begin{itemize}
	\item \textit{Chicago R Unconference} (\$200 Travel grant) by RStudio (2019).
			
	
	\item \textit{Dean's Competitive Fund for Promising Scholarship Award} (\$69,000) by Harvard University (2017-2018).
		
	
	\item {\it Summer Research Fellowship for Students and Teachers} by Indian Academy of Science (2008).
	
	
	\item {\it Young Scientist Fellowship} by Indian Institute of Science, Bangalore (2005-6).
	
	
	\item {\it Dhirubhai Ambani Scholarship} by Reliance Industries Limited (2005-8). 	
	\end{itemize}
	
	\heading{\textbf{Advising}}
	
	\begin{itemize}
	\item Undergraduate thesis/Internship/Exchange students: Carlotta Cogoni (2013-14); Marta Cal\`{o}, Federico Fornasier (2014-15); Jens Melsbach (2015-16); Micaela Maria Zucchelli (2017-18); Stephanie Campbell (2017-18); Tom Osborn (2018-2019).
	\end{itemize}
	
	\heading{\textbf{Chaired Symposia}}

    \begin{itemize}
	\item \textit{The Role of Reasoning in Moral Decision Making}. European Society for Cognitive Psychology, Tenerife, Spain, September, 2019.
    
	
	\item \textit{Inferring Moral Character from Moral Judgments}. Society for Personality and Social Psychology conference, Atlanta, Georgia, USA, March 3, 2018.
	
	
	\item \textit{Nice and Right: Clarifying the Relations Between Empathy and Moral Judgment}. International Conference for Psychological Science, Amsterdam, The Netherlands, March 24, 2015.
	\heading{\textbf{Invited Talks \& Colloquia}}
	\item \textit{Reasoning supports utilitarian resolutions to sacrificial moral dilemmas}. 21st meeting of the European Society for Cognitive Psychology, Tenerife, Spain, September 28, 2019.
	

	\item \textit{Reasoning supports utilitarian resolutions to sacrificial moral dilemmas}. EASP Meeting: Cognitive Conflicts: Taking a Cognitive Perspective on Social Phenomena, Tübingen, Germany, July 5, 2019.
	

    \item \textit{Reputational benefits of third-party compensation}. 31st Annual Human Behavior \& Evolution Society meeting, Boston, MA, USA, May 29, 2019.
	

	\item \textit{Reasoning supports utilitarian resolutions to sacrificial moral dilemmas}. 31st APS Annual Convention, Washington, D.C., USA, May 24, 2019.
	

	\item \textit{Reasoning and affect: Individual differences in moral conflict resolution}. Scalable Cooperation Group, MIT Media Lab, Cambridge, MA, USA, May 17, 2019.
	

	\item \textit{Reasoning supports utilitarian resolutions to moral dilemmas across diverse measures}. Social Perception and Evaluation Lab, New York University, New York City, NY, USA, March 29, 2019.
	

	\item \textit{ggstatsplot: R package for ggplot2 Based Plots with Statistical Details}. Boston useR Event, Boston, MA, USA, March 26, 2019.
	

	\item \textit{ggstatsplot: R package for ggplot2 Based Plots with Statistical Details}. Harvard Psych Methods Dinners, Harvard University, Cambridge, MA, USA, February 5, 2019.
	

	\item \textit{Reasoning supports utilitarian resolutions to moral dilemmas across diverse measures}. 'Reasoning in Moral Thought and Action' Workshop, Boston College, Boston, MA, USA, January 30, 2019.
		
	
	\item \textit{Reputational benefits of third-party compensation over punishment}. Social Psychology Lunch talk series, Harvard University, Cambridge, MA, USA, November 27, 2018.
		

%	\item \textit{I think, therefore I forgive: The role of reasoning in forgiving accidental harms}. Cikara lab meeting, Harvard University, Cambridge, MA, USA, November 19, 2018.
%		

	\item \textit{Reputational benefits of third-party compensation over punishment}. Boston Judgment and Decision Making Day, Boston College, Boston, MA, USA, November 2, 2018.
		

	\item \textit{ggstatsplot: R package for ggplot2 Based Plots with Statistical Details}. Social Cognitive Development Lab, Yale University, New Haven, CT, USA, September 19, 2018.
	

	\item \textit{Reputational benefits of third-party compensation over punishment}. Boston Area Moral Cognition Group meeting, Boston, MA, USA, March 22, 2018.
	

	\item \textit{Moral learning privileges model-free update}. European Society for Cognitive and Affective Neuroscience meeting, Leiden University Medical Center, Leiden, Netherlands, July 21, 2018.
	
	
	\item \textit{Moral learning privileges model-free update}. New England Research on Decision-Making conference, Harvard University, Cambridge, MA, USA, June 11, 2018.
	

	\item \textit{ggstatsplot: R package for ggplot2 Based Plots with Statistical Details}. Saxe Lab meeting talk, Massachusetts Institute of Technology, Cambridge, MA, USA, June 31, 2018.
	
		
	\item \textit{Reputational benefits of third-party compensation over punishment}. Boston Area Moral Cognition Group meeting, Boston, MA, USA, March 22, 2018.
	

	\item \textit{The behavioral and neural basis of empathic blame}. Social Cognitive Science Brown Bag, Brown University, Providence, Rhode Island, USA, March 16, 2018.
		
	
	\item \textit{Reputational benefits of third-party compensation over punishment}. Society for Personality and Social Psychology conference 2018, Atlanta, Georgia, USA, March 3, 2018.
	
	
	\item \textit{The behavioral and neural basis of empathic blame}. The Cooperation Lab, Boston College, Boston, MA, USA, February 1, 2018.
	
	
	\item \textit{The behavioral and neural basis of empathic blame}. Social Brain Brown Bag talk, Dartmouth College, Hanover, New Hampshire, USA, January 25, 2018.
	
	
	\item \textit{Neuroanatomical basis of concern-based altruism in virtual environment}. Boston Judgment and Decision Making Day, Boston University, Boston, MA, USA, April 21, 2017.
	
	
    %	\item \textit{A multimodal investigation of moral decision making in harmful contexts}. Moral Psychology Research Lab, Harvard University, Cambridge, MA, USA, February 23, 2017.
    %	
    %	
    %	\item \textit{A multimodal investigation of moral decision making in harmful contexts}. Harvard Intergroup Neuroscience Lab, Harvard University, Cambridge, MA, USA, February 6, 2017.
%	

	\item \textit{A multimodal investigation of moral decision making in harmful contexts}. Department of Developmental Psychology and Educational Psychology, Ludwig-Maximilians-Universit{\"a}t, Munich, Germany, July 14, 2016.
	
		
	\item \textit{Divergent role of empathy for wrongness and blame: an fMRI investigation}. International Conference for Psychological Science, Amsterdam, The Netherlands, March 24, 2015.

    \end{itemize}
    
	\heading{\textbf{Peer Review}}
	
	\begin{itemize}
	
	\item \textcolor{red}{{\textbf{Editorial Board}}}, \textit{Journal of Cognition} (2017-2019), \textit{Thinking \& Reasoning} (2018-2019).
	
	
	\item \textcolor{red}{{\textbf{Review Editor}}}, \textit{Frontiers in Human Neuroscience} (2014-2019), \textit{Frontiers in Psychiatry} (2018-2019).
	
	
	\item \textcolor{red}{{\textbf{Ad Hoc Reviewer}}}\\
	\textcolor{blue}{{Journal articles}} (60):\\
	\textit{Social Cognitive \& Affective Neuroscience}(2), \textit{Scientific Reports}(5), \textit{Cognition}(2), \textit{Social Neuroscience}(1), 
	\textit{Neuropsychologia}(2), \textit{PLoS One}(4), \textit{Frontiers in Psychology}(7), \textit{Journal of Cognitive Psychology}(6), 
	\textit{Motivation and Emotion}(1), \textit{Thinking \& Reasoning}(3), \textit{Social Influence}(1), \textit{Cognition \& Emotion}(1),
	\textit{International Journal of Psychology}(1),  \textit{Psychiatry Research}(2),
	\textit{Journal of Health and Social Sciences}(1),
	\textit{Judgment and Decision Making}(1), \textit{Evolution and Human Behavior}(1),   \textit{Frontiers in Human Neuroscience}(2), 
	\textit{Ethics \& Behavior}(1), \textit{Personality and Mental Health}(1),  \textit{Current Directions in Psychological Science}(1),
	\textit{Journal of Neuropsychology}(1), \textit{Current Psychology}(1), \textit{Experimental Psychology}(1), \textit{Social Cognition}(1), 
	\textit{Emotion}(3), \textit{British Journal of Psychology}(1), \textit{Journal of Experimental Child Psychology}(1), \textit{Journal of Affective Disorders}(1), \textit{Journal of Experimental Psychology: Learning, Memory, and Cognition}(1), \textit{SAGE Open}(1), \textit{Research in Autism Spectrum Disorders}(1), \textit{Philosophical Psychology}(1)  
	
	\textcolor{blue}{Book chapters} (1):\\
	\textit{Springer}(1)
	
	\end{itemize}
	
	\heading{\textbf{Articles for General Audience}}
	
	\begin{itemize}
	
	\item Authored \href{https://drive.google.com/file/d/0B6_u70YpdJKnUUdoM2sydFRuTmNLaURUQ1FxaE9uTS1oVFRF/view?usp=sharing}{ten popular science articles} in eminent Marathi newspaper \textit{Loksatta} on various topics from science and technology (2010-2011). 
	
	\item Ananthanarayan, B., Choudhary, K., Mohapatra, L., \textbf{Patil, I.}, Rustagi, A., \& Shivaraj, K. (2007). \href{http://eprints.iisc.ernet.in/26860/1/current.pdf}{Observation of Exotic Heavy Baryons}, \textit{Current Science}, 93:451-452.
	
	\item Ananthanarayan, B., Choudhary, K., Mohapatra, L., \textbf{Patil, I.}, Rustagi, A., \& Shivaraj, K. (2007). \href{http://eprints.iisc.ernet.in/15428/1/Observation_of_oscillation.pdf}{Observation of Oscillation Phenomena in Heavy Mesons Systems}, \textit{Current Science}, 93:602-604.
	
	\end{itemize}
	
	\heading{\textbf{References}}
	
	\parbox[c]{2.2in}{
	Dr. Giorgia Silani,\\
	University of Vienna,\\
	Austria.\\
	{\tt \colorbox{yellow}{giorgia.silani@univie.ac.at}}
	}
	\parbox[c]{2.2in}{
	Dr. Fiery Cushman,\\
	Department of Psychology,\\
	Harvard University, USA.\\
	{\tt \colorbox{yellow}{cushman@fas.harvard.edu}}
	}
	\parbox[c]{2.2in}{
	Dr. Mina Cikara,\\
	Department of Psychology,\\
	Harvard University, USA.\\
	{\tt \colorbox{yellow}{mcikara@fas.harvard.edu}}
	}
	%\parbox[c]{2.2in}{
	%Ezequiel Gleichgerrcht, MD\\
	%Department of Neurology,\\
	%Medical University of S. Carolina.\\
	%{\tt \colorbox{yellow}{gleichge@musc.edu}}
	%}
	%\parbox[c]{2.2in}{
	%Prof. Claus Lamm,\\
	%University of Vienna,\\
	%Austria-1010.\\
	%{\tt \colorbox{yellow}{claus.lamm@univie.ac.at}}
	%}
	%\parbox[c]{2.4in}{
	%Prof. P.B.Vidyasagar,\\
	%%University of Pune,\\
	%India-411 007.\\
	%{\tt pbv@physics.unipune.ernet.in}
	%}
	%\parbox[c]{2.4in}{
	%Dr. G.R. Kulkarni,\\
	%Univ. of Pune, Pune,\\
	%Pin-411 007.\\
	%{\tt grk@physics.unipune.ernet.in}
	%}
	%\parbox[c]{1.8in}{
	%Dr. R.V.Dabhade,\\
	%Fergusson College,\\
	%Pune, India-411~004.\\
	%{\tt rakadabhade@yahoo.com}
	%}
	
	

    %\heading{Research/Reading Projects Undertaken}
    %\item {\it Effect of hypergravity treatment on amylase and total protein content in germinating rice seeds}, a 
    %graduate project under Prof. P. B. Vidyasagar, Dept. of Physics, University of Pune (2009-10).
    
    
    %\item {\it Special Theory of Relativity}, a reading project under Prof. J. Maharana, Institute of Physics (IoP), 
    %Bhubaneshwar (2008).
    
    
    %\item {\it Observation of exotic heavy baryons and oscillation phenomena in heavy meson systems}, a reading 
    %project under Prof. B. Ananthanarayan, Indian Institute of Science, Bangalore (2007).
    
    
    %\item {\it Theoretical studies of static friction at atomic scale for unlubricated BCC solids}, an undergraduate 
    %project under Dr. D. D. Choughule and Dr. R. V. Dabhade, Fergusson College, Pune (2006).



    %\heading{Summer/Winter Schools Attended}
    %\item Winter School in Astroparticle Physics, Cosmic Ray Laboratory, Tata Institute of 
    %Fundamental Research (TIFR), Ooty, 2008.



    %\item Summer School in Gravitation and Cosmology, Harish-Chandra Research Institute (HRI), 
    %Allahabad, 2008.

    %\item Summer School in Social Cognitive Neuroscience, SISSA, Trieste, 2013. 
    %\item Summer School in Mathematics, Sir Parashurambhao (S.~P.) College, Pune, 2007.	
	
	%\heading{Summer/Winter Schools Attended}
	
	%\item Summer School in Social Cognitive Neuroscience, SISSA, Trieste, 2013. 
	
	%\item Winter School in Astroparticle Physics, Cosmic Ray Laboratory, Tata Institute of 
	%Fundamental Research (TIFR), Ooty, 2008.
	
	%\item Summer School in Gravitation and Cosmology, Harish-Chandra Research Institute (HRI), 
	%Allahabad, 2008.
	
	%\item Summer School in Mathematics, Sir Parashurambhao (S.~P.) College, Pune, 2007.
	
	
	%\heading{Miscellaneous}
	
	%\item Organised a three-day seminar on theoretical physics, `\textit{Frontiers in Physics}'  at Fergusson College in 
	%collaboration with the Foundational Questions Institute (FQXi), USA and IUCAA, Pune (2008).
	
	%\item Organised a three-day environmental awareness program by Ecology Club at Fergusson College,  held 
	%at Fergusson College (2008).
	
	%\item One of the Coordinators of the Physics and Astronomy Club at Fergusson College which organized 
	%several science exhibitions and quizzes for school and college students (2005-08).
	
	%\item A Wikipedia contributor (Username:~{\ttfamily inderonline1988})
	
	%\heading{Hobbies}
	%Writing (popular science, science fiction, screen-writing); Reading (fiction/non-fiction); Cinema
	
	\end{document}