\documentclass[12pt]{article}
\usepackage{epsfig}
\usepackage{graphicx}
\usepackage{graphics}
\usepackage{amsmath}
\usepackage{mhchem}
\usepackage{amssymb}
\usepackage{amsthm}
\usepackage{mathrsfs}
\usepackage{amsfonts}
\usepackage{cite}
\usepackage{color}
\usepackage{pdfcolmk}
\usepackage{enumerate}
\usepackage{epsf}
\usepackage{latexsym}
\usepackage{shadow}
\usepackage{array}
\usepackage{multirow}
\usepackage{pifont}
%\usepackage[pdftex,bookmarks,colorlinks]{hyperref}
%\hypersetup{colorlinks=true}
%\hypersetup{colorlinks,%
 %          citecolor=green,%
  %          filecolor=green,%
   %         linkcolor=green,%
    %        urlcolor=green,%
      %     pdftex}
\sloppy
%\newcommand{\times}{\ding{52}}
\newcommand{\e}{\epsilon}
\newcommand{\f}{\frac}
\newcommand{\be}{\beta}
\newcommand{\Ga}{\Gamma}
\newcommand{\Om}{\Omega}
\newcommand{\pr}{\prime}
\newcommand{\pa}{\partial}
\newcommand{\ber}{\begin{eqnarray}}
\newcommand{\eer}{\end{eqnarray}}
\newcommand{\bea}{\begin{equation}}
\newcommand{\eea}{\end{equation}}
\newcommand{\ud}{\,\mathrm{d}}
\topmargin = 0.1cm
\textheight = 23.0cm
\renewcommand{\baselinestretch}{1.4}
\begin{document}
\begin{normalsize}
\thispagestyle{empty}
\begin{large}
\title{\ \bf Effect of hypergravity treatment on amylase and total protein content in germinating rice seeds}
\end{large}
\vspace{4cm}
\author {by\\\bf Indrajeet~S.~Patil$^\dagger$ \\
             M.Sc.-II, Department of Physics,\\
            University of Pune, Pune-411~007\\
	     \begin{small}
	     \centerline{\tt($^\dagger$E-mail:~patilindrajeet.science@gmail.com)}    \end{small}\\
\\
             \bf Graduate Thesis-2010
\\
\\
Under the Guidance of:\\
\textbf{ Prof. P.B.Vidyasagar$^\ddagger$},\\
Department of Physics,\\
University of Pune,
Pune-411~007\\
\begin{small}
\centerline{\tt($^\ddagger$E-mail:~pbv@physics.unipune.ernet.in)}	                                                       \end{small}
}
%\date{}
\maketitle
\begin{figure}
\centering
\includegraphics[height=2.9cm,width=2.9cm,angle=0]{uoplogo.ps}
 \end{figure}
\thispagestyle{empty}
\newpage
\thispagestyle{empty}
\clearpage
\vspace*{1.5in}
\begin{center}
To,\\
\textit{Aai ani Pappa},\\
for their unconditional love and support.
\end{center}


\newpage
\thispagestyle{empty}
\clearpage
\vspace*{0.5in}

\begin{center}
``\textit{Though I do not believe\\
that a plant will spring up\\
where no seed has been,\\
I have great faith in seed.\\
Convince me that you have a seed there,\\
and I am prepared to expect wonders.}'' 

 -- \textbf{Henry D. Thoreau}

\end{center}

\vspace*{0.5in}
\begin{center}
``\textit{I never have died,\\ 
But close I hide\\
In a plummy seed that\\ 
the wind has sown.\\
Patient I wait through the\\ 
long winter hours;\\
You will see me again-\\
I shall laugh at you then,\\
Out of the eyes of a \\
hundred flowers.}''
   
-- ``Talking in their sleep" by \textbf{Edith M. Thomas}
\end{center}

\vspace*{0.5in}

\begin{center}
``\textit{Lost in thought and lost in time,\\
while the seeds of life and the seeds of change are planted\ldots}'' 

 -- ``Coming back to life" by \textbf{Pink Floyd}

\end{center}

\newpage

\tableofcontents{}
\newpage
\section{Gravitational Biology}
\subsection{Gravity of the Matter}
The idea of gravity is now part of our canonical knowledge and it is sometimes hard to believe that it was only in 1665 that Sir Isaac Newton propounded the basics of gravity.  Before Newton, it was assumed that there are different laws of motion for the movement of earthly and heavenly bodies. Newton brought down this dichotomy in the laws of nature and showed that the laws that govern the motion of moon going round the earth are same as those governing a glass falling off the table. This unification was the harbinger of the later unification to come, like that between electricity and magnetism.\\
After conceiving the notion of gravity, Newton ventured to formulate a quantitative force law of gravitation\footnote{In 1916, Albert Einstein introduced a new law of gravitation that completely metamorphosed the way we look at gravity. Newton's law can be derived as the approximation of the Einstein's theory. For the study of gravitational biology, however, Newtonian theory is adequate.}. He proposed that every particle in this universe attracts every other particle with a force directly proportional to the product of their masses and inversely proportional to the square of the distance between them, \textit{i.e.}
\begin{equation}
\vec{F}_{12}=G\frac{m_{1}m_{2}}{r^{2}}\hat{r}_{12}
\end{equation}
Note that the force on 2 due to 1 has the same magnitude as the force on 1 due to 2, but have opposite directions, in accordance with Newton's third law. Another point to be noted is that the only attribute that a particle must possess in order to be amenable to gravity is mass\footnote{In Einstein's theory, mass and/or energy.}. Another feature of this force is that the force between two particles is independent of the presence of any other particle/s.\\
That more or less concludes our recapitulation of the laws of gravity!

\subsection{Gravitational acceleration}
As per the Newton's second law, the force acting on a particle is equal to change in its linear momentum and, equivalently, equal to mass times acceleration. Thus, the force of gravity produces acceleration in the bodies.\\
If we treat the earth as a spherical object, we can apply Newton's shell theorem to it, which says that a spherical shell of matter attracts a particle that is outside the shell as if all the shell's mass were concentrated at its center. Thus, if we treat earth to be an assemblage of a lot of spherical shells, then we can apply shell theorem to obtain force on particle of mass,
\begin{equation}
F=\frac{GMm}{r^{2}}=mg
\end{equation}
where $M$ is the mass of the earth and $g$ is the acceleration due to gravity. This quantity $mg$ is what we call \textit{weight}.\\
Thus,
\begin{equation}
g=\frac{GM}{r^{2}}
\end{equation}
The direction of the acceleration is radially inwards, towards the center of the earth, at all points in the space. Na\"\i vely, we would expect the value of $g$ to be same for all the particles at the same radial distance. But the value of the $g$ varies at the surface of the earth for three reasons:
\begin{enumerate}
\item The Earth is not spherical; it is an oblate spheroid, flattened more at its poles.
\item The density of the earth is not uniform.
\item The Earth is rotating.
\end{enumerate}
The average value of $g$ at the surface, often taken into calculations, is $9.8~m/s^{2}$.\\
Because of the rotation of the earth, there is difference between the gravitational force and weight of the body. These two quantities are the same in static-earth scenario. But, in the rotating earth scenario, because of the additional centrifugal force in the particle frame, its weight tends to be lesser than the actual force by the amount determined by the magnitude of the centrifugal force, given by $F=mr\omega^{2}$. Thus, the force is more when radius is more and thus, the centrifugal force is maximum at the equator and ~minimum at the poles. And, therefore, the difference between weight and the gravitational force is maximum at the equator, while zero at the poles (since $\omega=0$).\\
As seen from Eqn(3), we see that $g$ is dependent on the mass of the planet $M$ and, thus, in outer space, we would find variety of astronomical bodies with humongous variety in their values of $g$. Thus, it is convenient to introduce two new terms:\\
\\
\begin{tabular}{|c|c|}
\hline
Microgravity & if $g < 9.8 ~m/s^{2}$ (e.g Pluto has $g=0.66~ m/s^{2}$) \\
\hline
Hypergravity & if $g > 9.8~ m/s^{2}$ (\textit{\textit{e.g.}} Sun has $g=275~ m/s^{2}$) \\
\hline
\end{tabular}

\subsection{Weightlessness}
We have all seen breathtaking pictures of astronauts doing space-walks or floating in the space capsules. These are situations in which we see microgravity environments, weightless environment, to be precise. 
There is an easy way to experience microgravity or hypergravity: your good old elevator!\\
Before going to that, we should understand the basic principle based on which our weight is measured on the weighing machine. When we stand on the machine, there is an electrostatic repulsion between the atoms of our feet/footwear and atoms of the material making up the machine because they are pushed up-close each other. This is what we term as normal reaction\footnote{Reason why we just do not sink in the earth and make journey to the center of the earth is because of this normal reaction!}. The machine measures this normal reaction and this is your weight in newtons.\\
Now, consider an elevator, going up with an acceleration $a$, has a man of mass $m$ standing on a weighing machine, then the normal reaction will be an addition of the force with which the gravity pulls the man down and the force that pulls the elevator, and hence the weighing machine, up, \textit{i.e.} $mg + ma$. Thus, your apparent weight will increase and so would the acceleration. This is one way of achieving hypergravity conditions (since resultant acceleration, $a+g$, is greater than $g$).\\
Similarly, if the elevator is accelerating in downward direction, then the net force acting would be $mg - ma$. Thus, your apparent weight would be reduced. This is the microgravity surroundings. If the cable supporting lift gets cut off, the last experience our subject would have is that of weightlessness!\\
In case of satellites or spacecrafts circling about the earth, both the floor and the crew are falling freely and, thus, feel weightless.

\subsection{Effects of altered gravity on life}
There is an apocryphal that the idea of gravity occurred to Newton while he was sitting under apple tree and an apple fell on his head. We might wonder if Newton ever pondered over the possible effect that gravity might have on the growth and development of that very tree, or living beings in general.\\
The strength and direction of the force of gravity has been constant throughout the evolutionary history of all the species on this planet. It has remained the same ever since the life arose on this planet. And hence, in addition to getting adapted to the changing climate and habitat, all living beings have adapted to the omnipresent force of gravity. In fact, they have learned to harness it, \textit{\textit{e.g.}} \textit{Paramecium}, which feeds on aerobic bacteria, shows a negative gravitaxis, \textit{i.e.} the cell population swims mainly upwards, against the gravity vector, thus reaching oxygen saturated layers.\\
Since Gagarin's first space journey, the field of gravitational biology has been all the rage. In future (or in \textit{Star}-\textit{Trek}-like scenarios!), humans might have to live in space for a long period or might have to colonize some other planet which may have the value of gravitational acceleration lesser or greater than $g$. Even currently, humans have been staying in space for long periods, making the study of gravitational biology all the more vital. And, thus, it is necessary to study the effects of gravity on humans, plants, and animals. Research in these areas would also provide us with fascinating insights into how gravity has shaped our evolution on this planet and how it still governs some of the basic life processes. Understanding the physiological changes caused by long-duration microgravity and hypergravity remains a daunting challenge. \\
Here we give few illustrative examples of the kind of effects that are observed under non-$g$ graviy conditions. For a more detailed discussion, Ref.[3]

\subsubsection{Effect on Plants}
Plant growth and development are affected by a lot of different environmental abiotic factors such as light, temperature, and water supply. Immediately upon germination, another physical stimulus, gravity, strongly influences the growth of plant organs, root and shoot, in order to ensure their correct orientation in space and the survival of the young seedling. Since plants have evolved under the constant stimulus of gravity, its presence is one of the most important prerequisites for their growth and spatial orientation. The ability of plants to change their growth orientation in response to gradients in light and gravity maximizes their ability to obtain energy from light and moisture and nutrients from soil. Plants show two principal responses to gravity:
\begin{enumerate}
\item Gravimorphogenesis, which enables plants to orient their leaves to sunlight for photosynthesis and their roots to soil for anchoring and absorbing water and minerals.
\item To resist the gravitational force by constructing a robust body. This graviresistance has been studied by centrifugation and space experiments and is very distinct from gravitropism.
\end{enumerate}
Plant organs such as shoots, roots, tendrils, and runners often perform rhythmic movements often in a helical spiral fashion with a period that ranges typically from minutes to several hours. Circumnutation and winding in plants are universal growth movements that allow plants to survive despite their sessile nature. These are gravity-dependent morphogenetic phenomena in plants where endodermal cells act as gravisensors. The amplitude and frequency of circumnutations in sunflower hypocotyls decreases and rotational change in direction occurs often in microgravity. In hypergravity, the amplitude and period of nutations increases.

\subsubsection{On Invertebrates}
Sea urchin sperm are sensitive to small changes in gravitational forces. More importantly this sensitivity has an effect on the ability of the sperms to fertilize eggs. Hypergravity decreases the hatching rate of eggs of \textit{C. elegans}. Oocyte meiotic division for exclusion of polar bodies shortly after fertilization is the most susceptible aspect to hypergravity.\\
Experiments on \textit{Drosophila melanogaster} carried out in space and on ground indicate that behavioral responses that may be important in setting life-spans of organisms may still be readily susceptible to manipulation by external cues.\textit{ Drosophila melanogaster} adults exposed to microgravity markedly increase their motility.\\
\textit{Drosophila melanogaster} is able to sense the gravity vector showing a clear negative geotactic response. The altered behavioral response to microgravity may have something to do with this gravitactic response as very young flies, \textit{i.e.} the ones immediately hatched out from the pupal case, show higher responses both behavioral and of accelerated ageing. Male flies were found to be more active than females during hypergeotaxis. Exposing \textit{Drosophila} to $2.5-7 g$ for short periods increased the longevity of the male flies but not the female flies. Young flies are affected longer by changing gravitational field than mature flies. Activity level of the flies was stimulated more by microgravity than by hypergravity while there was almost no difference in continuous velocity in both microgravity as well as hypergravity. 

\subsubsection{Lower vertebrates}
Fish have proven the most suited vertebrates for research into gravitational effects. Fish use visual and vestibular\footnote{vestibular system is concerned with balance and sense of spatial orientation} cues for postural equilibrium maintenance and orientation as other vertebrates and invertebrates. They often show abnormal swimming behavior like downward or upward pitching, inward looping, spinning movements, etc. especially during the transition from normal to microgravity. This is similar to the space adaptation syndrome seen in humans.\\
 Larval \textit{cichlid} fish that were allowed to complete their development in hypergravity were not affected in their morphogenetic development and also in the onset and performance of their swimming behavior. However, as soon as the centrifuge was stopped, many of the young fish revealed looping responses and spinning movements as observed after the transfer from $1 g$ to microgravity conditions in fish grown at $1 g$ which normally disappears. Fertilized eggs sent into space showed that embryos can develop normally in weightlessness and that locomotion in weightlessness is strongly disturbed.\\
  Studies on \textit{Zebrafish} show that there probably exists a critical period for functional maturation of the vestibular system. Female frogs were sent into space and induced to shed eggs that were artificially inseminated. The eggs did not rotate and yet, surprisingly, the tadpoles emerged and appeared normal. After return to Earth within 2�3 days of hatching, the tadpoles metamorphosed and matured into normal frogs. However tadpoles raised in microgravity tended to remain underwater and also had smaller lungs than normal. The optomotor\footnote{an instrument to test visual systems in animals} responses were stronger for tadpoles raised in microgravity while the tadpoles raised in hypergravity showed weaker optomotor responses.

\subsubsection{On Mammals}
Exposure to microgravity elicited by parabolic flights induces decrease of abdominal aortic pressure (AAP) in anesthetized rats. Exposing rats to $2 g$ for two weeks impaired their spatial learning ability suggesting that a constant gravity is needed for spatial learning.\\
 Electrophysiological studies suggest that the hypothalamus might be the most sensitive region of the rat limbic system. Mammary metabolic activity in pregnant rats increased in microgravity but decreased in hypergravity showing an exponential increase with $g$-load and a continuum from micro- to hypergravity environments.\\
 Exposing rats to hypergravity from embryonic stage till the age of 14 weeks increased the cell size without affecting the mechanosensory transduction in the vestibular system. This suggests that possibly a gravity-dependent mechanism is present during a particular development stage.\\
Orientation of the head with respect to gravity plays an important role in orienting and tuning the vertical angular vestibulo-ocular reflex gain in monkeys. Renal blood flow and terminal aorta blood flow reduced significantly even more than the cardiac output. Marked effect of microgravity on cardiac rhythm responses to otolith stimulation were observed in rhesus monkeys.

\subsubsection{Humans}
An important physiological phenomenon occurring in microgravity is the shift in body fluids towards the cardiopulmonary compartment. More than half of the astronauts experience space sickness (in fact, a form of motion sickness), the signs and symptoms of which are stomach discomfort, nausea, pallor, cold sweating, and vomiting.\\
Lack of gravitational loading affects multiple physiological systems, especially fluid flow, balance, and support structures that are particularly vulnerable to change or injury during re-entry and renewed exposure to gravitational forces. Exposure to microgravity, besides affecting the neurovestibular and respiratory systems, greatly alters the dynamics of the circulation and leads to bone demineralization and muscle atrophy. When taken together, circulatory deconditioning and muscle atrophy lead to a reduced exercise capacity and intolerance.\\
Orthostatic intolerance\footnote{Orthostatic intolerance is a disorder of autonomic nervous system. The patients suffering from it usually manifest the disorder by a temporary loss of consciousness and posture, with rapid recovery, as well as remaining conscious during their loss of posture.} after spaceflight can be attributed to decreases of cardiac filling pressure and stroke volume during orthostatic stress due to decreased blood volume. It is generally believed that the above modifications are completely reversible upon re-entry to normal $1g$ conditions even if it is still a matter of debate whether this is really the case after very long space flights.

\newpage
\section{Seed and its functions}
\subsection{What is seed?}
A seed is a small embryonic plant enclosed in a covering called the seed coat, usually with some stored food. It is the product of the ripened ovule\footnote{Ovule is a part of the ovary of seed plants that contains the female germ cell and, after fertilization, becomes the seed.} of gymnosperm\footnote{Gymnosperm is a plant of a group that comprises those that have seeds unprotected by an ovary or fruit. These are plants in which the ovules are not enclosed by any ovary wall and remain exposed, both before and after fertilization. The seeds that develop post-fertilization are not covered,\textit{e.g.} \textit{cycus}, \textit{pinus}, \textit{ginkgo}, etc.} and angiosperm\footnote{Angiosperm a plant of large group that comprises those that have flowers and produce seeds enclosed within a carpel (the female reproductive organ of a flower), including herbaceous plants, shrubs, grasses, and most trees, \textit{\textit{e.g.}} \textit{Eucalyptus}, wheat, rice, etc. These can be further divided into monocots and dicots.} plants which occurs after fertilization and some growth within the mother plant. The formation of the seed completes the process of reproduction in seed plants (started with the development of flowers and pollination), with the embryo developed from the zygote and the seed coat from the integuments of the ovule. Seeds have been an important development in the reproduction and spread of flowering plants, relative to more primitive plants like mosses, ferns and liverworts, which do not have seeds and use other means to propagate themselves, \textit{e.g.} vegetative propagation. This can be seen by the success of seed plants (both gymnosperms and angiosperms) in dominating biological niches on land, from forests to grasslands both in hot and cold climates.\\
Seed production in natural plant populations vary widely from year-to-year in response to weather variables, insects, diseases, and internal cycles within the plants themselves.

\subsection{Structure}
Any typical seed has three fundamental parts (refer fig.(1) and (2)):
\begin{description}
\item[Seed coat] : Seed coat is really a fruit coat. In all grains, the seed coat is fused to the ovary wall. So, in fact, the grain is technically a fruit even though we call it a seed. The main function of the seed coat is to keep the seed in dehydrated state that it exists in and limit the exchange of oxygen with the surroundings. Thus, the seed coat plays a crucial role in the seed dormancy and in protecting the embryo from any mechanical damage.

\item[Endosperm] : While still developing in the ovary, the developing embryo is surrounded by a nutritive tissue, or endosperm. Depending on the presence or absence of endosperm in the seed, the seeds are classified in two classes,  \textit{viz.}
\begin{description}
\item[endospermic seeds]: in which the endosperm is retained until maturity and comprises a major portion of the seed bulk, \textit{e.g.} castor, coconut, etc.
\item[non-embryonic seeds]: in which the endosperm is consumed by the embryo to satiate its energy needs and virtually the entire seed is occupied by the embryo, \textit{e.g.} pea, groundnut, beans. In such seeds, the cotyledons of the embryo become filled with this stored food.
\end{description}
Endosperm holds a huge reserve of starch, which is nothing but a chain of sugar molecules. This starch is converted to sugar and is, thus, the source of energy during the seed germination. In seed development, endosperm development precedes embryo development. 

\item[Embryo] : It is the part that grows after seed germination and grows into a tree. It further consists of three parts,
\begin{enumerate}
\item \textbf{Cotyledon}: It is an embryonic leaf, one or more of which are the first leaves to appear from a germinating seed. It is the region through which the starch converted into sugar is supplied to the embryo during the germination. Depending on the number of leaves, the seeds are classified\footnote{As a rule of thumb, most of the grains are monocots, while most of the fruits are dicots.} into monocots (only one leaf), \textit{\textit{e.g.}} barley, wheat, etc. or dicots (two such leaves), \textit{\textit{e.g.}} apples, mangoes, grapes, etc. We would shortly discuss the differences in the seed structures in these two cases.
The endosperm and cotyledon characteristically contain a large amount of carbon, mineral elements, and hormones that support the growth and development of the seedling until it can establish itself as a photosynthetically competent plane.

\item \textbf{Epicotyl}: The embryonic stem above the point of attachment of the cotyledon(s) is the epicotyl. In most plants, the epicotyl will eventually develop into the leaves of the plant. 

\item \textbf{Radicle}: The embryonic stem below the point of attachment is the hypocotyl. It grows downward in the soil.
\end{enumerate}	
\end{description}

\subsubsection{Structure of monocot seeds}
Monocot seeds are prevalently endospermic but some, as in orchids, are non-endospermic too. The endosperm is bulky and stores food. The outer covering of endosperm separates the embryo by a proteinous layer called aleurone layer. The embryo is small and situated in a groove at one end of the endosperm. It consists of one large and shield shaped cotyledon known as scutellum and a short axis with a plumule\footnote{plumule is the baby shoot and it grows after the radicle.} and a radicle. The plumule and radicle are enclosed in sheaths which are called as coleoptiles and coleorhizae, respectively. 
\begin{figure}
\centering
 \includegraphics[width=8cm,height=10cm,angle=0]{dicot.ps}
 \caption{Seed structure of Barley, a monocot.}
\end{figure}

\subsubsection{Structure of dicot seeds}
The seed coat consists of two layers, \textit{viz.} outer testa and inner tegmen. The hilum is a scar on the seed coat through which the developing seeds were attached to the fruit. Within the seed coat is the embryo, consisting of an embryonic axis and two cotyledons. At the two ends of the embryonic axis are present the radicle and the plumule. In some of the seeds, the endosperm is not present in mature seeds.
\begin{figure}
\centering
 \includegraphics[width=8cm,height=10cm,angle=0]{monocot.ps}
 \caption{Seed structure of Corn, a diocot.}
\end{figure}


\subsection{Seed functions}
Seeds serve several functions for the plants that produce them. Key among these functions are nourishment of the
embryo, dispersal to a new location, and dormancy (discussed later) during unfavorable conditions. Seeds fundamentally are a means
of reproduction and most seeds are the product of sexual reproduction which produces a remixing of genetic material
and phenotype variability that natural selection acts on. 

\begin{enumerate}
\item \textbf{Embryo nourishment}: Seeds protect and nourish the embryo or young plant. Seeds usually give a seedling a faster start than a sporeling from a spore, because of the larger food reserves in the seed and the multicellularity of the enclosed embryo.

\item \textbf{Seed dispersal}: Unlike animals, plants are limited in their ability to seek out favorable conditions for life and growth. As a result, plants have evolved many ways to disperse their offspring by dispersing their seeds. A seed must somehow \textit{arrive} at a location and be there at a time favorable for germination and growth. 
\end{enumerate}

\newpage
\section{Seed development}
The seed, which is an embryo with two points of growth (one of which forms the stems the other the roots) is enclosed in a seed coat with some food reserves. Its development takes two separate paths for angiosperms and gymnosperms:

\subsection{In angiosperms}
\begin{figure}
\centering
 \includegraphics[width=8cm,height=10cm,angle=0]{doublef.ps}
 \caption{The process of double fertilization in angiosperms.}
\end{figure}
Angiosperm seeds consist of three genetically distinct constituents:\\
(1) the embryo formed from the zygote,\\
(2) the endosperm, which is normally triploid (containing three homologous sets of chromosomes, \textit{i.e.}),\\
(3) the seed coat from tissue derived from the maternal tissue of the ovule.\\
 In angiosperms, the process of seed development begins with double fertilization and involves the fusion of the egg and sperm nuclei into a zygote. The second part of this process is the fusion of the polar nuclei with a second sperm cell nucleus, thus forming a primary endosperm. Right after fertilization, the zygote is mostly inactive but the primary endosperm divides rapidly to form the endosperm tissue. This tissue becomes the food that the young plant will consume until the roots have developed after germination or it develops into a hard seed coat. The seed coat forms from the two integuments or outer layers of cells of the ovule, which derive from tissue from the mother plant. When the seed coat forms from only one layer it is also called the testa, though not all such testa are homologous from one species to the next.\\
The ovules develop into seeds, while the ripened ovary develops into a fruit.

\subsection{In gymnosperms}
\begin{figure}
\centering
 \includegraphics[width=8cm,height=10cm,angle=0]{gymn_seed.ps}
 \caption{A typical gymnosperm seed.}
\end{figure}

In gymnosperms, the two sperm cells transferred from the pollen do not develop seed by double fertilization but one sperm nucleus unites with the egg nucleus and the other sperm is not used. Sometimes each sperm fertilizes an egg cell and one zygote is then aborted or absorbed during early development. The seed is composed of the embryo (the result of fertilization) and tissue from the mother plant, which also form a cone around the seed in coniferous plants, like \textit{Pine} and \textit{Spruce}. The ovules after fertilization develop into the seeds.

\subsection{Hormonal changes}
Seed development is characterized by often dramatic changes in hormone levels. In most seeds, cytokinin\footnote{a plant hormone that promotes cell division.} levels are highest during the very early stages of embryo development when the rate of cell division is also highest. As the cytokinin level declines and the seed enter a period of rapid cell enlargement, both GA (gibberellic acid) and IAA (Indole-3-aceic acid) levels increase. In the early embryogenesis, there is little or no detectable ABA (abscisic acid). It is during the latter stages of embryo development, as GA and IAA levels begin to decline, that ABA levels begin to rise. ABA levels generally peak during the maturation stage when seed volume and dry weight also reach a maximum. Maturation of the embryo is characterized by cessation of embryo growth, accumulation of nutrient reserves, and the development of tolerance to desiccation. ABA serves to prevent vivipary, a precocious germination before the embryo reaches maturity or the seed is released from the fruit.     

\newpage
\section{Seed Dormancy}
Seed dormancy has two main functions: \\
(i) In synchronizing germination with the optimal conditions for survival of the resulting seedling\\
(ii) In spreading germination of a batch of seeds over time so that a catastrophe after germination (\textit{e.g.}, late frosts, drought, herbivory) does not result in the death of all offspring of a plant. \\
Seed dormancy is defined as a seed failing to germinate under environmental conditions optimal for germination, normally when the environment is at a suitable temperature with proper soil moisture. This true dormancy or innate dormancy is therefore caused by conditions within the seed that prevent germination. Thus, dormancy is a state of the seed, not of the environment.\\
 Induced dormancy, enforced dormancy, or seed quiescence occurs when a seed fails to germinate because the external environmental conditions are inappropriate for germination, mostly in response to conditions being too dark or light, too cold or hot, or too dry. 
 
\subsection{Types of dormancy}
Often seed dormancy is divided into four major categories: exogenous; endogenous; combinational; and secondary.
A more recent system distinguishes five classes of dormancy:morphological, physiological, morphophysiological,
physical and combinational dormancy.\\
Exogenous dormancy is caused by conditions outside the embryo including:
\begin{description}
\item[Physical dormancy] occurs when seeds are impermeable to water. At dormancy, a
specialized structure, the water gap, is disrupted in response to environmental cues, especially temperature, so
that water can enter the seed and germination can occur.

\item[Chemical dormancy] considers species that lack physiological dormancy, but where a chemical prevents
germination. This chemical can be leached out of the seed by rainwater or snow melt or be deactivated
somehow. Leaching of chemical inhibitors from the seed by rain water is often cited as an important cause of
dormancy release in seeds of desert plants, however little evidence exists to support this claim.
\end{description}

Endogenous dormancy is caused by conditions within the embryo itself, including:
\begin{description}
\item[Morphological dormancy] where germination is prevented due to morphological characteristics of the embryo. In
some species the embryo is just a mass of cells when seeds are dispersed, it is not differentiated. Before
germination can take place both differentiation and growth of the embryo have to occur. In other species the
embryo is differentiated but not fully grown (underdeveloped) at dispersal and embryo growth up to a species
specific length is required before germination can occur. 

\item[Morphophysiological dormancy] seeds with underdeveloped embryos, and which in addition have physiological
components to dormancy. These seeds therefore require a dormancy-breaking treatments as well as a period of
time to develop fully grown embryos. 

\item[Physiological dormancy] means that the embryo can, due to physiological causes, not generate enough power to
break through the seed coat, endosperm or other covering structures. Dormancy is typically broken at cool wet,
warm wet, or warm dry conditions. Abscisic acid is usually the growth inhibitor in seeds and its production can be
affected by light.

\item[Combinational dormancy] implies the seed or fruit coat is impermeable to water
and the embryo has physiological dormancy. Depending on the species physical dormancy can be broken before
or after physiological dormancy is broken.

\item[Secondary dormancy] is caused by conditions after the seed has been dispersed and occurs in some seeds when
non-dormant seed is exposed to conditions that are not favorable to germination, very often high temperatures.
The mechanisms of secondary dormancy are not yet fully understood.
\end{description}

The following types of seed dormancy do not involve seed dormancy strictly spoken, as lack of germination is
prevented by the environment not by characteristics of the seed itself:
\begin{description}
\item[Photodormancy] or light sensitivity affects germination of some seeds. These photoblastic seeds need a period of
darkness or light to germinate. In species with thin seed coats, light may be able to penetrate into the dormant
embryo. The presence of light or the absence of light may trigger the germination process, inhibiting germination
in some seeds buried too deeply or in others not buried in the soil.

\item[Thermodormancy] is seed sensitivity to heat or cold. Some seeds germinate
only at high temperatures ($30^{\circ}\mathrm{C}$). Many plants that have seed that germinate in early to mid summer have thermodormancy and germinate only when the soil temperature is warm. Other seeds need cool soils to germinate,
while others are inhibited when soil temperatures are too warm. 
\end{description}
Not all seeds undergo a period of dormancy. Seeds of some mangroves are viviparous, they begin to germinate while
still attached to the parent. The large, heavy root allows the seed to penetrate into the ground when it falls. Many
garden plants have seeds that will germinate readily as soon as they have water and are warm enough, though their
wild ancestors may have had dormancy, these cultivated plants lack seed dormancy. After many generations of
selective pressure by plant breeders and gardeners dormancy has been selected out.

\subsection{Seed origin and evolution}
The origin of seed plants is a problem that still remains unsolved. However, more and more data tends to place this origin in the middle Devonian. The description in 2004 of the proto-seed \textit{Runcaria heinzelinii} in the Givetian of Belgium is an indication of that ancient origin of seed-plants. As with modern ferns, most land plants before this time reproduced by sending spores into the air, that would land and become whole new plants.\\
The first \textit{true} seeds are described from the upper Devonian, which is probably the theater of their true first evolutionary radiation. The seed plants progressively became one of the major elements of nearly all ecosystems.

\newpage
\section{Requirements for seed germination}
Germination\footnote{We note that the process in which fungus emerges from a spore and begins growth is also called as germination.} is the growth of an embryonic plant contained within a seed; it results in the formation of the seedling. The seed of a higher plant is a small package produced in a fruit or cone after the union of male and female sex cells. All fully developed seeds contain an embryo and, in most plant species, some store of food reserves, wrapped in a seed coat.\\
 Some plants produce varying numbers of seeds that lack embryos, these are called empty seeds and never germinate. Most seeds go through a period of quiescence where there is no active growth; during this time the seed can be safely transported to a new location and/or survive adverse climate conditions until circumstances are favourable for growth.\\
Seed germination depends on both internal and external conditions. The most important factors include temperature, water, oxygen and sometimes light or darkness. Various plants require different variables for successful seed germination, often this depends on the individual seed variety and is closely linked to the ecological conditions of a plant's natural habitat.

\begin{description}
\item[Water] is required for germination. Mature seeds are often extremely dry and need to take in significant amounts of water, relative to the dry weight of the seed, before cellular metabolism and growth can resume. Most seeds need enough water to moisten the seeds but not enough to soak them. The uptake of water by seeds is called imbibition, which leads to the swelling and the breaking of the seed coat. When seeds are formed, most plants store a food reserve with the seed, such as starch, proteins, or oils. This food reserve provides nourishment to the growing embryo. When the seed imbibes water, hydrolytic enzymes are activated which break down these stored food resources into metabolically useful chemicals. After the seedling emerges from the seed coat and starts growing roots and leaves, the seedling's food reserves are typically exhausted; at this point photosynthesis provides the energy needed for continued growth and the seedling now requires a continuous supply of water, nutrients, and light.

\item[Oxygen] is required by the germinating seed for metabolism. Oxygen is used in aerobic respiration, the main source of the seedling's energy until it grows leaves. Oxygen is an atmospheric gas that is found in soil pore spaces; if a seed is buried too deeply within the soil or the soil is waterlogged, the seed can be oxygen starved. Some seeds have impermeable seed coats that prevent oxygen from entering the seed, causing a type of physical dormancy which is broken when the seed coat is worn away enough to allow gas exchange and water uptake from the environment.

\item[Temperature] affects cellular metabolic and growth rates. Seeds from different species and even seeds from the same plant germinate over a wide range of temperatures. Seeds often have a temperature range within which they will germinate, and they will not do so above or below this range. Many seeds germinate at temperatures slightly above room-temperature ($16-24^{\circ} \mathrm{C}$), while others germinate just above freezing and few others germinate only in response to alternations in temperature between warm and cool. Some seeds germinate when the soil is cool ($-2 - 4^{\circ} \mathrm{C}$), and some when the soil is warm ($24-32^{\circ} \mathrm{C}$). Some seeds require exposure to cold temperatures (vernalization) to break dormancy. Seeds in a dormant state will not germinate even if conditions are favorable. Seeds that are dependent on temperature to end dormancy have a type of physiological dormancy. For example, seeds requiring the cold of winter are inhibited from germinating until they take in water in the fall and experience cooler temperatures. $4^{\circ} \mathrm{C}$ is cool enough to end dormancy for most cool dormant seeds, but some groups need conditions cooler than $-5^{\circ} \mathrm{C}$. Some seeds will only germinate after hot temperatures during a forest fire which cracks their seed coats; this is a type of physical dormancy.

\item[Light or darkness] can be an environmental trigger for germination and is a type of physiological dormancy. Most seeds are not affected by light or darkness, but many seeds, including species found in forest settings, will not germinate until an opening in the canopy allows sufficient light for growth of the seedling. 
\end{description}

\subsection{Seedling establishment}
In some definitions, the appearance of the radicle marks the end of germination and the beginning of \textit{establishment}, a period that ends when the seedling has exhausted the food reserves stored in the seed. Germination and establishment as an independent organism are critical phases in the life of a plant when they are the most vulnerable to injury, disease, and water stress, exactly like a new-born baby! The mortality between dispersal of seeds and completion of establishment can be so high that many species have adapted to produce huge numbers of seeds.

\newpage
\section{Molecular biological understanding of germination}
\begin{figure}
\centering
 \includegraphics[width=8cm,height=10cm,angle=0]{seedgerm.ps}
%\label{Figure-2}
 \caption{Steps in the process of seed germination.}
\end{figure}

In the life cycle of plants, seed germination period might be called as infancy period. In this period, the seed embryo grows, when hydrated, with the help of food storage that it carries in the form of starch. Seed germination is said to have occurred when growth of the radicle bursts the seed coat and protrudes as a young root. As dicots utilize their endosperm by the time of maturation, they do not have any nutrient reserves. As we are studying the germination of wheat and rice, which is a monocot, we will study the biochemistry of monocot seed germination in depth. The treatment here heavily relies on the study that was carried out on, another monocot, barley by plant physiologists. They were trying to study how to improve yield of sugar from barley. The main steps involved in the process of seed germination are as follows:
\subsection{Stage-1: Imbibition}
When non-dormant seeds are soaked in water, it(water) ruptures the testa and gushes into the interiors of seed. This is called as imbibition. In this process, water penetrates the seed coat and begins to soften the hard, dry tissues inside. The water uptake causes the grain to swell up. The seed/fruit coat usually splits open allowing water to enter even faster, leading to a positive feedback. The water begins to activate the biochemistry of the dormant embryo. By this time, the levels of both gibberellin\footnote{Gibberellins (GAs) are plant hormones that regulate growth and influence various developmental processes, including stem elongation, germination, dormancy, flowering, sex expression, enzyme induction, and leaf and fruit senescence.} and active IAA\footnote{a plant hormone that promotes elongation of the embryonic axis} increase. The gibberellins appear to be released by the hydrated embryo from a preformed GA pool. Gibberellins are responsible\footnote{This can be confirmed by a simple experiment. Cereal grains such as barley may be transected to produce two half-seeds; such that one half-seed contains embryo and the other half-seed does not. When imbibed, the embryo-containing half-seed will proceed to secrete $\alpha$-amylase, an enzyme to break down starch into sugar, and would thus initiate the process of nutrient mobilization. The half-seed without embryo, of course, can not germinate but neither does it produce elevated levels of $\alpha$-amylase or any other hydrolytic enzyme required for germination. Treatment of the embryoless half-seed with GA, however, will stimulate the half-seed to produce high levels of $\alpha$-amyalse!} for the mobilization of nutrient reserves stored in the endosperm. This begins with gibberellins getting dissolved into water and reaching the cytoplasm of aleurone cell. 

\subsection{Stage-2: Transcription of DNA}
The GA crosses into the cytoplasm of the aleurone cells and turns on certain genes in the nuclear DNA. DNA is, of course, the hereditary molecule and contains the instructions for making every protein needed for the survival. The precise mechanism of how GA turns on the DNA is unknown at present. It is clear, however, that the mode of action is to turn on just certain genes in the DNA. The genes that are turned on are transcribed. The information archived in the DNA is precious, so the aleurone cells make a disposable RNA copy of the gene that is turned on. This disposable copy of the information, a kind of blueprint, is often called messenger RNA. The process of making this RNA copy is called transcription. 

\begin{figure}
\centering
\includegraphics[width=8cm,height=10cm,angle=0]{centraldogma.ps}
%\label{Figure-2}
\caption{Central dogma of molecular biology.}
\end{figure}

\subsection{Stage-3: Translation of DNA}
The RNA that was made in the transcription process is transported into the cytoplasm of the aleurone cells. In the cytoplasm, the messenger RNA joins up with a ribosome to begin the process of making a protein. This process is often called protein synthesis or translation. In this process, the ribosome examines the information held in the sequence of bases in the RNA. Transfer RNAs (t-RNA), charged with particular amino acids, are moved into the position specified by the instructions in the messenger RNA, and the amino acids are joined in a proper sequence by the ribosome through a peptide linkage. The sequence of amino acids determines the properties of the protein being assembled, since protein is nothing but a long chain of amino acids. In this case, the critical protein made with the information held in the RNA is $\alpha$-amylase. This protein turns out to be an enzyme of great importance.\\ 
The amino acid building blocks for the amylase come from some other biochemistry in the aleurone cells. This biochemistry causes the storage proteins in the aleurone cells to be digested by hydrolytic enzymes. The hydrolysis (the chemical breakdown of a compound due to reaction with water) is accelerated by enzymes known as \textit{proteases}\footnote{Protease is an enzyme which breaks down the proteins and peptides}. These enzymes increase the rate at which the storage protein is cut into individual amino acids. The amino acids released by the hydrolysis are then free to be reassembled by the ribosomes into the structure of amylase\footnote{The same thing happens in people. You are not what you eat! You do not slowly turn into a chicken(!) by eating chicken. The protein in your chicken is digested into amino acids. Those amino acids are then reassembled into human proteins. Since the instructions for reassembling the amino acids come from your human DNA, the proteins produced are human, not chicken! In a similar way, barley aleurone storage proteins are digested and amylase is made from the released amino acids.}.

\subsection{Stage-4: Sugar production}
The amylase is secreted (transported out) from the aleurone cells into the endosperm. Amylase is not just any old protein. It happens to be an enzyme. The amylases speed up the hydrolysis of starch into its component sugar units. Also, the protease converts an inactive $\beta$-amylase to the active form and $\alpha$-and $\beta$-amylase together digest starch to glucose and maltose. These sugars are moved to embryo through the region of cotyledon and embryo�s metabolic demands are met with.\\
Since we want to study the effect of gravity on the content of amylases, we will study the process of hydrolysis of starch in detail in Appendix-B

\newpage
\section{Tetrazolium test of viability}
Seeds, though dormant, are \textit{viable} for certain period, \textit{i.e.} post-treatment they can be made to germinate, even the deeply dormant seeds. But, the age of seed affects its health and germination ability. Since the seed has a living embryo, over period of time cells die and can't be replaced. If the seed is not viable, it will not germinate under any condition. The period of viability varies from variety to variety of seeds. Some maple species have seeds that need to sprout within two weeks of being dispersed, or they die, while some seeds of Lotus plants are known to be up to 2000 years old and still can be germinated. 
 Thus, it is necessary to conduct a test to first check for the validity of the seeds to be used in the experiment. This test is known as Tetrazolium (TZ, for short) test, because a chemical named 2,3,5-triphenyltetrazolium chloride (TTC) is used. This chemical is a redox indicator and, thus, can be used to indicate cellular respiration. 

\subsection{Procedure}
Tetrazolium solution is prepared by preparing a 0.1-1 \% solution of TTC. The pH of the solution has to be 7 for proper staining. Solutions of pH 4 or lower will not stain even viable embryos and solution of pH more than 8 will result in too intense staining. The seeds stain faster at higher temperatures. But, TTC is somewhat unstable at high temperature and, therefore, the solution should be kept at room temperature.\\
The seeds are soaked for 24 hours so that the embryo becomes metabolically active. Then seeds are cut through their embryos and kept in the solution. The white compound is enzymatically reduced to red TPF (1,3,5-triphenylformazon)  in living tissues due to the activity of various dehydrogenases\footnote{enzymes important in oxidation of organic compounds and, thus, in cellular metabolism}, while it remains as white TTC in areas of necrosis(premature death of cells and living tissue) since these enzymes have been either denatured or degraded.  
\begin{figure}
\centering
 \includegraphics[width=10cm,height=15cm,angle=0]{TZ.ps}
 \caption{Chemical reaction describing TZ-test}
\end{figure}

By TZ test, it was found that the seeds are viable, as seen in the picture. About 95\% rice seeds were viable.
\begin{figure}
\centering
 \includegraphics[width=8cm,height=15cm,angle=0]{TZ_result.ps}
\caption{Wheat seeds pass the TZ-test. The red portions in the picture show the portions of the seed where metabolism is active. Similar results were obtained for rice seeds, though photograph was not taken, due to unavailability of a camera.}
\end{figure}

\newpage
\section{Experimental Procedure}
In the current experiment, we want to study the effect of hypergravity treatment on the contents of total protein and amylases inside the germinating seeds. For this, experimental procedure is followed in the following order:
\begin{enumerate}
\item Germinate the seeds and grow them on agar-agar gel instantly for \textit{control} (non-hypergravity-treated, \textit{i.e.}).
\item Give hypergravity treatment to 24-hour imbibed seeds and then grow them on agar-agar gel.
\item Prepare enzyme assays for control and hypergravity-treated seeds at specified hours. 
\item Prepare DNSA reagent\footnote{A reagent is a substance or compound that is added to a system in order to bring about a chemical reaction or is added to see if a reaction occurs. Such a reaction is used to confirm the presence of another substance.\\
In our experiment, DNSA and Bradford reagents are used to confirm presence of maltose and BSA respectively.} for calculating $\beta$-amylase content.
\item Obtain a standard graph for maltose by plotting maltose concentration vs. absorbance.
\item Prepare Bradford reagent for calculating total protein content.
\item Obtain a standard graph for BSA by plotting BSA (bovine serum albumin) concentration vs. absorbance. 
\item Using spectrophotometer, measure absorbance of the samples and, by comparing it with standard graphs, compute the concentrations of amylases and total protein.  
\end{enumerate}

\subsection{Preparation of germinating seeds}
Rice (species: \textit{Basmati}) seeds are soaked for 24 hours in water. Before soaking, the seeds are cleansed with 0.5 \% solution of fungicide. After soaking, the seeds start germinating. For few days, seeds do not require any nutrients, since they harvest energy from starch hydrolysis. Thus, they can be grown, post-soaking, on 0.8\% agar-agar gel for further readings.

\subsection{Hypergravity treatment}
To take readings for control (\textit{i.e.} not treated with hypergravity), seeds are directly put on agar-agar gel after soaking. Other seeds are given treatments with different values of $g$, \textit{viz.} $500g$, $1000g$, $1500g$, $2000g$, $2500g$, $5000g$ for 10 min. at room temperature using centrifuge. These are then sown on agar-agar gel. Both beakers of control and treated seeds are kept inside seed germinator. Readings are taken at $0^{th}$, $4^{th}$, $8^{th}$, $12^{th}$, $24^{th}$, $48^{th}$ hours. The time is taken to be zero at the completion of 24-hour soaking.\\
To do this, seeds are taken in centrifugation tubes in 1 $ml$ of water. It has been shown that if seeds are not put in water, hypergravity does not have any effect on dry seeds.
\begin{center}
\begin{tabular}{|c|c|c|c|c|c|c|}
\hline
$g-$value ($\downarrow$)/ Hour ($\rightarrow$) & 0 & 4 & 8 & 12 & 24 & 48 \\
\hline
control & $\times$ & $\times$ & $\times$ & $\times$ & $\times$ & $\times$ \\
\hline
500 & $\times$ & $\times$ & $\times$ & $\times$ & $\times$ & $\times$ \\
\hline
1000 & $\times$ & $\times$ & $\times$ & $\times$ & $\times$ & $\times$ \\
\hline
1500 & $\times$ & $\times$ & $\times$ & $\times$ & $\times$ & $\times$ \\
\hline
2000 & $\times$ & $\times$ & $\times$ & $\times$ & $\times$ & $\times$ \\ 
\hline
2500 & $\times$ & $\times$ & $\times$ & $\times$ & $\times$ & $\times$ \\
\hline
5000 & $\times$ & $\times$ & $\times$ & $\times$ & $\times$ & $\times$ \\
\hline
\end{tabular}
\end{center}
\vspace*{0.1in}
Thus, in total, we need to take 42 samples and for each of them, we need to prepare enzyme assay and Bradford protein assay to estimate amylase and protein quantities.

\subsection{Preparing solutions required for extraction amylases}
In order to compute the concentration of amylase ($\alpha$ as well as $\beta$), we use DNSA reagent. We first plot the standard graph for maltose, \textit{viz.} concentration of maltose vs. absorbance. Thus, by comparing the value of absorbance of the sample, we can directly compute the concentration of maltose and, thus, indirectly compute concentration of amylase. 
\begin{description}
\item[Starch solution] : Weigh $1 gm$ of starch and put it in $100ml$ of boiling water to it, continue boiling for 2-3 minutes. Let the solution cool down and filter it. Use the filtrate as substrate solution. Store it always at $30^{\circ}\mathrm{C}$ and not in fridge.

\item[Sodium Phosphate buffer ($0.05M, pH-7$)] :\\
As we know, buffer solution is a solution which resits change in its $pH$ value when acid or alkali is added to it. The sodium phosphate buffer is prepared as follows,\\
 $(i)$	Molecular weight of $Na_{2}HPO_{4}$ (more basic) is $141.96 gm$. $1M$ is $141.96 gm$ in $1000ml$ distilled water. Therefore, $0.05M$ is $7.098 gm$ in $1000ml$ distilled water, \textit{i.e.} $0.7098 gm$ in $100ml$.\\
$(ii)$	Molecular weight of $NaH_{2}PO_{4}$ (less acidic) is $156.01 gm$. $1M$ is $156.01 gm$ in $1000ml$ distilled water. Therefore, $0.05M$ is $7.8005 gm$ in $1000ml$ distilled water, \textit{i.e.} $0.7801gm$ in $100ml$.\\
Add $(i)$ slowly in $(ii)$ until $pH$ of the solution becomes 7.

\item[DNSA Reagent] : Take $40 gm$ sodium potassium tartarate in $40ml$ of slightly warm distilled water and dissolve it. Add $1 gm NaOH$ and mix it by glass rod. Then add $1 gm$ DNSA and mix it by glass rod. Make the total volume $100ml$.
\end{description}

\subsection{Preparing enzyme assay}
For every reading, we need to prepare an enzyme assay\footnote{An assay is a procedure in molecular biology for testing or measuring the activity of a biochemical in an organism or organic sample.} in order to compute the content of enzyme. It is prepared in following way:\\
$(1)$ Macerate $0.5 gm$ of seeds with $5 ml$ of phosphate buffer in a chilled mortar using liquid nitrogen and pestle. Mix it with shaker for $2-3$ minutes.\\
$(2)$ Centrifuge at $10000 g$ in a refrigerated centrifuge at $4^{\circ}\mathrm{C}$ for 15 minutes. Keep enzyme preparation at $-40^{\circ}\mathrm{C}$.\\
$(3)$ Repeat same procedure for extracting enzyme at $4^{th}$, $8^{th}$, $12^{th}$, $24^{th}$, and $48^{th}$ hours after soaking.\\
$(4)$ Aliquot the enzyme preparation in number of centrifugation tubes.

\subsection{Amylase content using DNSA reagent}
\begin{description}
\item[For standard] :\\
In order to find out the amount of amylase, we must have plotted the maltose standard graph. This is because $\beta-$amylase acts on starch and produces maltose, while $\alpha-$amylase produces maltose and glucose and, thus, we can infer the amount of total amylase by computing the amount of maltose based on the standard graph.\\
For this purpsoe, we use DNSA reagent. 3,5-Dinitrosalicylic acid is an aromatic compound that reacts with reducing sugars\footnote{A sugar is only a reducing sugar if it has an open chain with an aldehyde or a ketone group. Reducing monosaccharides include glucose, fructose, glyceraldehyde and galactose.} and other reducing molecules to form 3-amino-5-nitrosalicylic acid, which absorbs light strongly at 540$nm$.\\
Note that instead of directly measuring concentrations of amylase, we \textit{indirectly} infer its quantity from the amount of maltose.\\  
Procedure to plot the standard graph is as follows:\\
$(1)$ Weigh $30 mg$ of maltose and mix it in $1ml$ of distilled water. (Concentration is $30 mg/ml$.)\\
$(2)$ Take $300 \mu l$ of above solution and add $2700 \mu l$ of distilled water. (Concentration is $3 mg/ml$.)\\
$(3)$ Take 10 test-tubes and add stock as per given in the table below and dilute it with distilled water:
\\
\begin{center}
\begin{tabular}{|c|c|c|}
\hline
Maltose conc. ($mg/ml$) & Maltose quantity ($\mu l$) & Distilled water ($\mu l$)\\
\hline \hline
0.3 & 50 & 450 \\
\hline
0.6 & 100 & 400 \\
\hline
0.9 & 150 & 350 \\
\hline
1.2 & 200 & 300 \\
\hline
1.5 & 250 & 250 \\
\hline
1.8 & 300 & 200 \\
\hline
2.1 & 350 & 150 \\
\hline
2.4 & 400 & 100 \\
\hline
2.7 & 450 & 50 \\
\hline
3.0 & 500 & 0 \\
\hline
\end{tabular} 
\end{center}
\vspace*{0.1in}
$(4)$ Add $2ml$ of DNSA in each tube. \\
$(5)$ Keep all the tubes in boiling water for 5 min. Change in color must appear.\\
$(6)$ After 5 min., take out all tubes and cool them.\\
$(7)$ After cooling, add $1.5ml$ of distilled water to each tube so that the total volume becomes $3 ml$. Take $2 ml$ of water and $1ml$ of DNSA to prepare base.\\
$(8)$ Measure optical density of all solutions at $540 nm$ using spectrophotometer\footnote{While taking these reading make sure that you clean cuvettes every, say, three readings. This is because maltose is a sticky substance and after few readings, a layer forms on the inner surface of the cuvette and absorbance readings increase irrespective of the change in the quantity of maltose. If available, use digital ultrasonic cleaner rather than ethanol or acetone.} for absorbance readings. \\
$(8)$ Draw a graph of concentration versus absorbance. This will be the maltose standard graph.

\item[For sample]:\\
$(1)$ Pipette out $0.2 ml$ of starch solution and $0.2 ml$ of distilled water, and $0.1 ml$ of appropriately diluted enzyme preparation into tubes.\\
$(2)$ Incubate all the tubes at $37^{\circ}\mathrm{C}$ for 15 ~minutes and then stop reaction by adding $1 ml$ of DNSA. Keep both the tubes in boiling water bath for 5 min.\\
$(3)$ Cool and make the volume upto $3 ml$ by adding $1.5 ml$ of distilled water. Prepare blanks by taking a solution of $2 ml$ distilled water plus $1 ml$ of DNSA.\\
$(4)$ Record the absorbance of the samples using DNSA reagent as blank at $540 nm$.
\end{description}
 
\subsection{Preparation of Bradford's Reagent}
The Bradford protein assay is a spectroscopic analytical procedure used to measure the concentration of protein in a solution. It is subjective, \textit{i.e.} dependent on the amino acid composition of the measured protein.\\
The Bradford assay, a colorimetric protein assay, is based on an absorbance shift of the dye Coomassie Brilliant Blue G-250 under acidic conditions when a redder form of the dye is converted into a blue form on binding to protein. During the formation of this complex, two types of bond interaction take place: the red form of Coomassie dye first donates its free electron to the ionizable groups on the protein, which causes a disruption of the protein's native state, consequently exposing its hydrophobic pockets. These pockets on the protein's tertiary structure bind non-covalently to the non-polar region of the dye \textit{via} van der Waals forces, positioning the positive amine groups in proximity with the negative charge of the dye. The bind is further strengthened by the ionic interaction between the two. Binding of the protein stabilizes the blue form of Coomassie dye, thus the amount of complex present in a solution is a measure for the protein concentration by use of an absorbance reading.\\
The bound form of the dye has an absorption spectrum maximum historically held to be at $595nm$. The cationic (unbound) forms are green or red while binding of the dye to protein stabilizes the blue ionic form. The increase of absorbance at $595nm$ is proportional to the amount of dye, thus to the amount (concentration) of protein present in the sample.\\ 
The Bradford assay is linear over a short range, typically from $2 \mu g/ml$ to $120 \mu g/ml$, making dilutions necessary. Care should be taken that all the utensils used are extremely clean, since Bradford is prone to detergent and this can give misleading readings.\\
Bradford's reagent is prepared as follows:\\
$(1)$ Take $100 mg$ of CBB G-250 in a brown bottle.\\
$(2)$ Add $50 ml$ of chilled ethanol and stir it for $30-40 min.$ using magnetic stirrer.\\
$(3)$ After this, continue stirring and add $100 ml$ phosphoric acid drop by drop using burette in it and continue stirring for $15 min.$\\
$(4)$ Continue stirring and add $350 ml$ of distilled water drop by drop.\\
$(5)$ Stop stirring and filter solution with Whatman filter paper. The filtrate will serve as Bradford Reagent.

\subsection{Total Protein content using Bradford protein assay}
BSA is also commonly used to determine the quantity of other proteins, by comparing an unknown quantity of protein to known amounts of BSA. BSA is used because of its stability, its lack of effect in many biochemical reactions, and its low cost since large quantities of it can be readily purified from bovine blood, a by-product of the cattle industry.
\begin{description}
\item[For standard] :\\
$(1)$	Take $10 mg$ BSA in $1 ml$ distilled water.\\
$(2)$	Take $100 \mu l$ \textit{i.e.} $0.1 ml$ above solution (step-1) and add $0.9 ml$ distilled water ($1 mg/ml$).\\
$(3)$	Take $500 \mu l$ \textit{i.e.} $0.5 ml$ above solution (step-2) and add $4.5 ml$ distilled water ($0.1 mg/ml$).\\
$(4)$	Take 10 centrifuge tubes. Take $10\mu l$, $20\mu l$,\ldots ,$100\mu l$ BSA and add $90\mu l$, $80\mu l$,\ldots , $0\mu l$ distilled water in it.
\\
\begin{center}
\begin{tabular}{|c|c|c|}
\hline
BSA conc. ($\mu g/ml$) & $0.1 mg/ml$ BSA quantity & Distilled water ($\mu l$)\\
\hline \hline
10 & 10 & 90 \\
\hline
20 & 20 & 80\\
\hline
30 & 30 & 70 \\
\hline
40 & 40 & 60 \\
\hline
50 & 50 & 50 \\
\hline
60 & 60 & 40 \\
\hline
70 & 70 & 30 \\
\hline
80 & 80 & 20 \\
\hline
90 & 90 & 10 \\
\hline
100 & 100 & 0 \\
\hline
\end{tabular}
\end{center}
\vspace*{0.1in}
$(5)$	Take 2 tubes filled with $100 \mu l$ distilled water as blank and add $1 ml$ Bradford reagent in each.\\
$(6)$	Also add Bradford reagent in all tubes in step 4. After adding Bradford reagent, shake tubes immediately for 2-3 seconds.\\
$(7)$	Keep tubes at room temperature for 5-7 min.\\
$(8)$	Take absorbance at $590nm$ of samples made in step-4 with blank from step-5.

\item[For sample] :\\ 
$(1)$	Take $10 \mu l$ of sample (\textit{i.e.} enzyme preparation). Add $90 \mu l$ distilled water in it.\\
$(2)$	Add $1 ml$ Bradford reagent in it.\\
$(3)$	Take $100\mu l$ of distilled water and $1ml$ of Bradford reagent to make blank.\\
$(4)$	Record absorbance at $590nm$.\\
\end{description}

\newpage
\section{Observations for amylase estimation}
In this section, we would assimilate all the experimentally obtained data related to $\beta$-amylase estimation.
\subsection{Readings to plot maltose standard graph}
These are observed absorbances for different concentrated solutions of maltose:\\
\begin{center}
\begin{tabular}{|c|c|}
\hline
Maltose conc. ($mg/ml$) & Absorbance\\
\hline \hline
0.3 & 0.3437 \\
\hline
0.6 & 0.7802\\
\hline
0.9 & 1.3302\\
\hline
1.2 & 1.7775\\
\hline
1.5 & 2.2059\\
\hline
1.8 & 2.4089 \\
\hline
2.1 & 3.2927\\
\hline
2.4 & 3.6878\\
\hline
2.7 & 3.7025\\
\hline
3.0 & 4.1945\\
\hline
\end{tabular}
\end{center}
\vspace*{0.1in}
The graph corresponding to these readings has been plotted in Fig.(9). 
\begin{figure}
\centering
 \includegraphics[width=20cm,height=20cm,angle=90]{maltose_std.ps}
 \caption{This is the standard graph for maltose, plotted with absorbances for different concentrations of BSA in the solution. The red line shows the linear fit with equation of line being $y=0.6625x-0.625$.}
\end{figure}
\subsection{Readings to estimate maltose and, thus, amylase content}
Here we will first take the absorbance of samples: 
\subsection{For control}
\begin{tabular}{|c|c|c|c|c|} 
\hline
Hour & Sample no. & Absorbance ($A$) & Average ($\bar{A}$) & S.D. ($\sigma_{A}$) \\
\hline \hline
\multirow{3}{*}{0} & 1 & 1.0409 & \multirow{3}{*}{1.0402} & \multirow{3}{*}{0.0023}\\ \cline{2-3}
& 2 & 1.0421 &\\ \cline{2-3}
& 3 & 1.0376 &\\  \cline{2-3}
\hline
\multirow{3}{*}{4} & 1 & 0.9253 & \multirow{3}{*}{1.1037} & \multirow{3}{*}{0.2523}\\ \cline{2-3}
& 2 & 1.2821 &\\ \cline{2-3}
& 3 & - &\\  \cline{2-3}
\hline
\multirow{3}{*}{8} & 1 & 1.2352 & \multirow{3}{*}{1.2413} & \multirow{3}{*}{0.0086}\\ \cline{2-3}
& 2 & 1.2474 &\\ \cline{2-3}
& 3 & - &\\  \cline{2-3}
\hline
\multirow{3}{*}{12} & 1 & 0.9417 & \multirow{3}{*}{0.8950} & \multirow{3}{*}{0.0456}\\ \cline{2-3}
& 2 & 0.8926 &\\ \cline{2-3}
& 3 & 0.8506 &\\ \cline{2-3} 
\hline
\multirow{3}{*}{24} & 1 & 1.4776 & \multirow{3}{*}{1.4424} & \multirow{3}{*}{0.1100}\\ \cline{2-3}
& 2 & 1.3191 &\\ \cline{2-3}
& 3 & 1.5304 &\\ \cline{2-3} 
\hline
\multirow{3}{*}{48} & 1 & 1.0103 & \multirow{3}{*}{1.0226} & \multirow{3}{*}{0.0322}\\ \cline{2-3}
& 2 & 0.9984 &\\ \cline{2-3}
& 3 & 1.0592 &\\ \cline{2-3} 
\hline
\end{tabular}

\subsection{For $500g$}
\begin{tabular}{|c|c|c|c|c|}
\hline
Hour & Sample no. & Absorbance ($A$) & Average ($\bar{A}$) & S.D. ($\sigma_{A}$) \\
\hline \hline
\multirow{3}{*}{0} & 1 & 1.4840 & \multirow{3}{*}{1.4616} & \multirow{3}{*}{0.0328}\\ \cline{2-3}
& 2 & 1.4239 &\\ \cline{2-3}
& 3 & 1.4769 &\\  \cline{2-3}
\hline
\multirow{3}{*}{4} & 1 & 1.6637 & \multirow{3}{*}{1.6296} & \multirow{3}{*}{0.0970}\\ \cline{2-3}
& 2 & 1.5202 &\\ \cline{2-3}
& 3 & 1.7049 &\\  \cline{2-3}
\hline
\multirow{3}{*}{8} & 1 & 1.3104 & \multirow{3}{*}{1.3314} & \multirow{3}{*}{0.0297}\\ \cline{2-3}
& 2 & 1.3524 &\\ \cline{2-3}
& 3 & - &\\  \cline{2-3}
\hline
\multirow{3}{*}{12} & 1 & 2.0670 & \multirow{3}{*}{2.0297} & \multirow{3}{*}{0.0449}\\ \cline{2-3}
& 2 & 1.9772 &\\ \cline{2-3}
& 3 & 2.0248 &\\ \cline{2-3}
 \hline
\multirow{3}{*}{24} & 1 & 2.2195 & \multirow{3}{*}{2.0808} & \multirow{3}{*}{0.2098}\\ \cline{2-3}
& 2 & 1.8395 &\\ \cline{2-3}
& 3 & 2.1834 &\\  \cline{2-3}
\hline
\multirow{3}{*}{48} & 1 & 0.5129 & \multirow{3}{*}{0.4839}& \multirow{3}{*}{0.0258}\\ \cline{2-3}
& 2 & 0.4636 &\\ \cline{2-3}
& 3 & 0.4751 &\\  \cline{2-3}
\hline
\end{tabular}

\subsection{For $1000g$}
\begin{tabular}{|c|c|c|c|c|} \hline
Hour & Sample no. & Absorbance ($A$) & Average ($\bar{A}$) & S.D. ($\sigma_{A}$) \\
\hline \hline
\multirow{3}{*}{0} & 1 & 0.8083 & \multirow{3}{*}{0.9520} & \multirow{3}{*}{0.2032}\\ \cline{2-3}
& 2 & 1.0956 &\\ \cline{2-3}
& 3 & - &\\  \cline{2-3}
\hline
\multirow{3}{*}{4} & 1 & 1.0570 & \multirow{3}{*}{1.0503}& \multirow{3}{*}{0.0074}\\ \cline{2-3}
& 2 & 1.0423 &\\ \cline{2-3}
& 3 & 1.0515 &\\ \cline{2-3}
\hline
\multirow{3}{*}{8} & 1 & 0.9567 & \multirow{3}{*}{0.9605}& \multirow{3}{*}{0.0209}\\ \cline{2-3}
& 2 & 0.9417 &\\ \cline{2-3}
& 3 & 0.9830 &\\  \cline{2-3}
\hline
\multirow{3}{*}{12} & 1 & 0.8719 & \multirow{3}{*}{0.9199} & \multirow{3}{*}{0.0421}\\ \cline{2-3}
& 2 & 0.9503 &\\ \cline{2-3}
& 3 & 0.9375 &\\  \cline{2-3}
\hline
\multirow{3}{*}{24} & 1 & 1.4381 & \multirow{3}{*}{1.4978}& \multirow{3}{*}{0.0762}\\ \cline{2-3}
& 2 & 1.4717 &\\ \cline{2-3}
& 3 & 1.5836 &\\  \cline{2-3}
\hline
\multirow{3}{*}{48} & 1 & 1.0877 & \multirow{3}{*}{1.0079}& \multirow{3}{*}{0.1498}\\ \cline{2-3}
& 2 & 0.8352 &\\ \cline{2-3}
& 3 & 1.1010 &\\  \cline{2-3}
\hline
\end{tabular}

\subsection{For $1500g$}
\begin{tabular}{|c|c|c|c|c|} \hline
Hour & Sample no. & Absorbance ($A$) & Average ($\bar{A}$) & S.D. ($\sigma_{A}$) \\
\hline \hline
\multirow{3}{*}{0} & 1 & 0.5697 & \multirow{3}{*}{0.5573} & \multirow{3}{*}{0.0118}\\ \cline{2-3}
& 2 & 0.5462 &\\ \cline{2-3}
& 3 & 0.5560 &\\ \cline{2-3} 
\hline
\multirow{3}{*}{4} & 1 & 0.5949 & \multirow{3}{*}{0.5435}& \multirow{3}{*}{0.0521}\\ \cline{2-3}
& 2 & 0.5450 &\\ \cline{2-3}
& 3 & 0.4907 &\\ \cline{2-3} 
\hline
\multirow{3}{*}{8} & 1 & 0.9298 & \multirow{3}{*}{1.0561}& \multirow{3}{*}{0.1252}\\ \cline{2-3}
& 2 & 1.0583 &\\ \cline{2-3}
& 3 & 1.1802 &\\  \cline{2-3}
\hline
\multirow{3}{*}{12} & 1 & 1.4136 & \multirow{3}{*}{1.3962} & \multirow{3}{*}{0.0700}\\ \cline{2-3}
& 2 & 1.3192 &\\ \cline{2-3}
& 3 & 1.4559 &\\  \cline{2-3}
\hline
\multirow{3}{*}{24} & 1 & 1.0481 & \multirow{3}{*}{1.0054}& \multirow{3}{*}{0.0949}\\ \cline{2-3}
& 2 & 0.8967 &\\ \cline{2-3}
& 3 & 1.0714 &\\  \cline{2-3}
\hline
\multirow{3}{*}{48} & 1 & 1.0239 & \multirow{3}{*}{1.1259}& \multirow{3}{*}{0.0887}\\ \cline{2-3}
& 2 & 1.1690 &\\ \cline{2-3}
& 3 & 1.1849 &\\  \cline{2-3}
\hline
\end{tabular}

\subsection{For $2000g$}
\begin{tabular}{|c|c|c|c|c|} \hline
Hour & Sample no. & Absorbance ($A$) & Average ($\bar{A}$) & S.D. ($\sigma_{A}$) \\
\hline \hline
\multirow{3}{*}{0} & 1 & 1.2736 & \multirow{3}{*}{1.2638} & \multirow{3}{*}{0.0139}\\ \cline{2-3}
& 2 & 1.2540 &\\ \cline{2-3}
& 3 & - &\\ \cline{2-3}
 \hline
\multirow{3}{*}{4} & 1 &  1.4017 & \multirow{3}{*}{1.4081}& \multirow{3}{*}{0.0136}\\ \cline{2-3}
& 2 & 1.4231 &\\ \cline{2-3}
& 3 & 1.3995 &\\ \cline{2-3}
\hline
\multirow{3}{*}{8} & 1 & 1.5294 & \multirow{3}{*}{1.5222}& \multirow{3}{*}{0.0072}\\ \cline{2-3}
& 2 & 1.5151 &\\ \cline{2-3}
& 3 & 1.5221  &\\ \cline{2-3}
\hline
\multirow{3}{*}{12} & 1 & 1.6822 & \multirow{3}{*}{1.6833}& \multirow{3}{*}{0.0015}\\ \cline{2-3}
& 2 & 1.6843 &\\ \cline{2-3}
& 3 & - &\\ \cline{2-3}
 \hline
\multirow{3}{*}{24} & 1 & 2.1492 & \multirow{3}{*}{2.0851}& \multirow{3}{*}{0.0907}\\ \cline{2-3}
& 2 & 2.0209 &\\ \cline{2-3}
& 3 & - &\\ \cline{2-3}
 \hline
\multirow{3}{*}{48} & 1 & 1.2866 & \multirow{3}{*}{1.2472}& \multirow{3}{*}{0.1070}\\ \cline{2-3}
& 2 & 1.2685 &\\ \cline{2-3}
& 3 & 1.2676 &\\ \hline \cline{2-3}
\end{tabular}

\subsection{For $2500g$}
\begin{tabular}{|c|c|c|c|c|} \hline
Hour & Sample no. & Absorbance ($A$) & Average ($\bar{A}$) & S.D. ($\sigma_{A}$) \\
\hline \hline
\multirow{3}{*}{0} & 1 & - & \multirow{3}{*}{0.8699} & \multirow{3}{*}{0.1440}\\ \cline{2-3}
& 2 & 0.9717 &\\ \cline{2-3}
& 3 & 0.7680 &\\  \cline{2-3}
\hline
\multirow{3}{*}{4} & 1 & 0.9315 & \multirow{3}{*}{0.9341}& \multirow{3}{*}{0.0063}\\ \cline{2-3}
& 2 & 0.9412 &\\ \cline{2-3}
& 3 & 0.9295 &\\ \cline{2-3}
\hline
\multirow{3}{*}{8} & 1 & 1.0009 & \multirow{3}{*}{0.8219}& \multirow{3}{*}{0.1629}\\ \cline{2-3}
& 2 & 0.6825 &\\ \cline{2-3}
& 3 & 0.7822 &\\ \cline{2-3}
\hline
\multirow{3}{*}{12} & 1 & 1.0866 & \multirow{3}{*}{0.9459} & \multirow{3}{*}{0.1990}\\ \cline{2-3}
& 2 & 0.8052 &\\ \cline{2-3}
& 3 & - &\\  \cline{2-3}
\hline
\multirow{3}{*}{24} & 1 & 0.9889 & \multirow{3}{*}{0.9957}& \multirow{3}{*}{0.0660}\\ \cline{2-3}
& 2 & 1.0648 &\\ \cline{2-3}
& 3 & 0.09334 &\\  \cline{2-3}
\hline
\multirow{3}{*}{48} & 1 & 0.6830 & \multirow{3}{*}{0.7688}& \multirow{3}{*}{0.1199}\\ \cline{2-3}
& 2 & 0.9058 &\\ \cline{2-3}
& 3 & 0.7176 &\\  \cline{2-3} 
\hline
\end{tabular}

\subsection{For $5000g$}
\begin{tabular}{|c|c|c|c|c|} \hline
Hour & Sample no. & Absorbance ($A$) & Average ($\bar{A}$) & S.D. ($\sigma_{A}$) \\
\hline \hline
\multirow{3}{*}{0} & 1 & 1.0230 & \multirow{3}{*}{0.9614} & \multirow{3}{*}{0.0612}\\ \cline{2-3}
& 2 & 0.9606 &\\ \cline{2-3}
& 3 & 0.9007 &\\  \cline{2-3}
\hline
\multirow{3}{*}{4} & 1 & 1.2366 & \multirow{3}{*}{1.1472}& \multirow{3}{*}{0.1265}\\ \cline{2-3}
& 2 & 1.0577 &\\ \cline{2-3}
& 3 & - &\\ \cline{2-3}
\hline
\multirow{3}{*}{8} & 1 & 1.3416 & \multirow{3}{*}{1.2278}& \multirow{3}{*}{0.1490}\\ \cline{2-3}
& 2 & 1.0591 &\\ \cline{2-3}
& 3 & 1.2827 &\\ \cline{2-3}
\hline
\multirow{3}{*}{12} & 1 & 1.3581 & \multirow{3}{*}{1.4642} & \multirow{3}{*}{0.0924}\\ \cline{2-3}
& 2 & 1.5079 &\\ \cline{2-3}
& 3 & 1.5266 &\\  \cline{2-3}
\hline
\multirow{3}{*}{24} & 1 & 1.1875 & \multirow{3}{*}{1.1702}& \multirow{3}{*}{0.0834}\\ \cline{2-3}
& 2 & 1.2436 &\\ \cline{2-3}
& 3 & 1.0795 &\\  \cline{2-3}
\hline
\multirow{3}{*}{48} & 1 & 1.7594 & \multirow{3}{*}{1.7054}& \multirow{3}{*}{0.2278}\\ \cline{2-3}
& 2 & 1.4555 &\\ \cline{2-3}
& 3 & 1.9013 &\\  \cline{2-3} 
\hline
\end{tabular}

\subsection{Estimated Concentrations of maltose from standard graph}
From the standard graph of maltose, we know that the equation of line is,
\begin{equation}
y= 0.6625x+0.0625
\end{equation} 
Thus,
\begin{equation}
\boxed{~\text{Conc. of maltose in} (mg/ml)= 0.6625~\text{absorbance}+0.0625~} 
\end{equation}
Also, we know that if $x=Ay+B$, then $\sigma_{x}=A\sigma_{y}$. This helps us to convert error bars of BSA graph to total protein graph.\\
Thus,
\begin{equation}
\boxed{~\sigma_{\text{conc.}}=0.6625~\sigma_{\text{absorbance}}~} 
\end{equation}
\\
The estimated concentration of maltose (along with error in it) as tabulated below:
\begin{center}
\begin{tabular}{|c|c|c|c|c|}
\hline
\multirow{2}{*}{Hr.} & \multicolumn{4}{|c|}{Conc. of maltose ($mg/ml$)}\\ \cline{2-5}
& control & $500g$ & $1000g$ & $1500g$ \\
\hline \hline
0  & 0.7100 $\pm$ 0.0014 & 0.9724 $\pm$ 0.0204 & 0.6551 $\pm$ 0.0127 & 0.4095 $\pm$  0.0007\\
\hline
4  & 0.7496 $\pm$ 0.1571 & 1.0769 $\pm$ 0.0060 & 0.7163 $\pm$ 0.0005 & 0.4008 $\pm$ 0.0033\\
\hline
8  & 0.8352 $\pm$ 0.0054 & 0.8913 $\pm$ 0.0185 & 0.6604 $\pm$ 0.0013 & 0.7199 $\pm$ 0.0079\\
\hline
12 & 0.6196 $\pm$ 0.0284 & 1.3260 $\pm$ 0.0280 & 0.6351 $\pm$ 0.0026 & 0.9316 $\pm$ 0.0044\\
\hline
24 & 0.9604 $\pm$ 0.0068 & 1.3578 $\pm$ 0.1306 & 0.9949 $\pm$ 0.0048 & 0.6884 $\pm$ 0.0059\\
\hline
48 & 0.6991 $\pm$ 0.0200 & 0.3637 $\pm$ 0.0156 & 0.6899 $\pm$ 0.0094 & 0.7634 $\pm$ 0.0055\\
\hline
\end{tabular}
\end{center}
\vspace{0.2in}	

\begin{center}
\begin{tabular}{|c|c|c|c|}
\hline
\multirow{2}{*}{Hr.} & \multicolumn{3}{|c|}{Conc. of maltose ($mg/ml$)}\\ \cline{2-4}
&  $2000g$ & $2500g$ & $5000g$ \\
\hline \hline
0  & 0.8492 $\pm$ 0.0009  & 0.6040 $\pm$ 0.0090  & 0.6610 $\pm$ 0.0038  \\
\hline
4  & 0.9390 $\pm$ 0.0009 & 0.6440 $\pm$ 0.0004 & 0.7766 $\pm$ 0.0079\\
\hline
8  & 1.0101 $\pm$ 0.0005 & 0.5741 $\pm$ 0.0102 & 0.8268 $\pm$ 0.0009\\
\hline
12 & 1.1104 $\pm$ 0.0001 & 0.6513 $\pm$ 0.0124 & 0.9752 $\pm$ 0.0058 \\
\hline
24 &  1.3605 $\pm$ 0.0057 & 0.6823 $\pm$ 0.0041 & 0.7910 $\pm$ 0.0052\\
\hline
48 & 0.8389 $\pm$ 0.0067 & 0.5411 $\pm$ 0.0075 & 1.1241 $\pm$ 0.0142\\
\hline
\end{tabular}
\end{center}
		
These readings have been plotted with and without error bars in figures(10) and (11) respectively.
\begin{figure}
\centering
 \includegraphics[width=20cm,height=20cm,angle=90]{amylase_conc_noerr.ps}
 \caption{This is the concentrations of maltose and thus, amylase (in $mg/ml$), for different post-imbibition hours and for different values of hypergravity. These are estimated from the standard graph for BSA. The error bars are not shown. The time at which compare readings is $24^{th}$ hour.}
\end{figure}
\\
\begin{figure}
\centering
 \includegraphics[width=20cm,height=20cm,angle=90]{amylase_conc_err.ps}
 \caption{This is the concentrations of total maltose and, thus, amylase (in $mg/ml$), for different post-imbibition hours and for different values of hypergravity. These are estimated from the standard graph for maltose. The error bars are shown. The time at which compare readings is $24^{th}$ hour.}
\end{figure}
	
\newpage
\section{Observations for total protein content estimation}
In this section, we would assimilate all the experimentally obtained data related to total protein estimation.

\begin{figure}
\centering
 \includegraphics[width=20cm,height=20cm,angle=90]{BSA_std.ps}
 \caption{This is the standard graph for BSA, plotted with absorbances for different concentrations of BSA in the solution. The red line shows the linear fit with equation of line being $y=9656.6x-18.5$.}
\end{figure}

\subsection{Readings to plot BSA standard graph}
These are observed ansorbances for solution of different concentrations of bsa.\\

\begin{center}
\begin{tabular}{|c|c|c|c|} \hline
Conc. ($\mu g/ml$) of BSA & Sample no. & Absorbance ($A$) & Average ($\bar{A}$) \\
\hline \hline
\multirow{3}{*}{10} & 1 & 0.0360 & \multirow{3}{*}{0.0274} \\ \cline{2-3}
& 2 & 0.0140  &\\ \cline{2-3}
& 3 & 0.0322 &\\  \cline{2-3}
\hline
\multirow{3}{*}{30} & 1 & 0.0445 & \multirow{3}{*}{0.0537}\\ \cline{2-3}
& 2 & 0.0555 &\\ \cline{2-3}
& 3 & 0.0612 &\\  \cline{2-3}
\hline
\multirow{3}{*}{50} & 1 & 0.0768 & \multirow{3}{*}{0.0782} \\ \cline{2-3}
& 2 & 0.0777 &\\ \cline{2-3}
& 3 & 0.0801 &\\  \cline{2-3}
\hline
\multirow{3}{*}{70} & 1 & 0.0909 & \multirow{3}{*}{0.0855}\\ \cline{2-3}
& 2 & 0.0839 &\\ \cline{2-3}
& 3 & 0.0818 &\\  \cline{2-3}
\hline
\end{tabular}
\end{center}
\vspace*{0.1in}
The graph of these readings is shown in Fig.(12)

\subsection{Total protein content measurements}
Here we will first take the absorbance of samples and then from the standard graph of BSA, we will compute the concentration of BSA. 
\subsubsection{For control}
\begin{tabular}{|c|c|c|c|c|} \hline
Hour & Sample no. & Absorbance ($A$) & Average ($\bar{A}$) & S.D. ($\sigma_{A}$) \\
\hline \hline
\multirow{3}{*}{0} & 1 & 0.0066 & \multirow{3}{*}{0.0064} & \multirow{3}{*}{0.0002}\\ \cline{2-3}
& 2 & 0.0062 &\\ \cline{2-3}
& 3 & - &\\  \cline{2-3}
\hline
\multirow{3}{*}{4} & 1 & 0.0314 & \multirow{3}{*}{0.0261}& \multirow{3}{*}{0.0050}\\ \cline{2-3}
& 2 & 0.0255 &\\ \cline{2-3}
& 3 & 0.0214 &\\  \cline{2-3}
\hline
\multirow{3}{*}{8} & 1 & 0.0122 & \multirow{3}{*}{0.0151}& \multirow{3}{*}{0.0041}\\ \cline{2-3}
& 2 & 0.0180 &\\ \cline{2-3}
& 3 & - &\\  \cline{2-3}
\hline
\multirow{3}{*}{12} & 1 & 0.0599 & \multirow{3}{*}{0.0504} & \multirow{3}{*}{0.0134}\\ \cline{2-3}
& 2 & - &\\ \cline{2-3}
& 3 & 0.0409 &\\  \cline{2-3}
\hline
\multirow{3}{*}{24} & 1 & 0.0565 & \multirow{3}{*}{0.0659}& \multirow{3}{*}{0.0133}\\ \cline{2-3}
& 2 & - &\\ \cline{2-3}
& 3 & 0.0753 &\\  \cline{2-3}
\hline
\multirow{3}{*}{48} & 1 & 0.0275 & \multirow{3}{*}{0.0359}& \multirow{3}{*}{0.0080}\\ \cline{2-3}
& 2 & 0.0433 &\\ \cline{2-3}
& 3 & 0.0369 &\\  \cline{2-3}
\hline
\end{tabular}

\subsection{For $500g$}
\begin{tabular}{|c|c|c|c|c|} \hline
Hour & Sample no. & Absorbance ($A$) & Average ($\bar{A}$) & S.D. ($\sigma_{A}$) \\
\hline \hline
\multirow{3}{*}{0} & 1 & - & \multirow{3}{*}{0.0482} & \multirow{3}{*}{0.0098}\\ \cline{2-3}
& 2 & 0.0413 &\\ \cline{2-3}
& 3 & 0.0551 &\\  \cline{2-3}
\hline
\multirow{3}{*}{4} & 1 & 0.0505 & \multirow{3}{*}{0.0505}& \multirow{3}{*}{0.0148}\\ \cline{2-3}
& 2 & 0.0652 &\\ \cline{2-3}
& 3 & 0.0357 &\\  \cline{2-3}
\hline
\multirow{3}{*}{8} & 1 & 0.0489 & \multirow{3}{*}{0.0419}& \multirow{3}{*}{0.0099}\\ \cline{2-3}
& 2 & 0.0349 &\\ \cline{2-3}
& 3 & - &\\  \cline{2-3}
\hline
\multirow{3}{*}{12} & 1 & 0.0645 & \multirow{3}{*}{0.0751} & \multirow{3}{*}{0.0150}\\ \cline{2-3}
& 2 & 0.0858 &\\ \cline{2-3}
& 3 & - &\\  \cline{2-3}
\hline
\multirow{3}{*}{24} & 1 & 0.0699 & \multirow{3}{*}{0.0674}& \multirow{3}{*}{0.0035}\\ \cline{2-3}
& 2 & 0.0649 &\\ \cline{2-3}
& 3 & - &\\  \cline{2-3}
\hline
\multirow{3}{*}{48} & 1 & - & \multirow{3}{*}{0.0478}& \multirow{3}{*}{0.0176}\\ \cline{2-3}
& 2 & 0.0514 &\\ \cline{2-3}
& 3 & 0.0441 &\\  \cline{2-3}
\hline
\end{tabular}

\subsection{For $1000g$}
\begin{tabular}{|c|c|c|c|c|} \hline
Hour & Sample no. & Absorbance ($A$) & Average ($\bar{A}$) & S.D. ($\sigma_{A}$) \\
\hline \hline
\multirow{3}{*}{0} & 1 & - & \multirow{3}{*}{0.0195}& \multirow{3}{*}{0.0054} \\ \cline{2-3}
& 2 & 0.0233 &\\ \cline{2-3}
& 3 & 0.0156 &\\  \cline{2-3}
\hline
\multirow{3}{*}{4} & 1 & 0.0379 & \multirow{3}{*}{0.0553}& \multirow{3}{*}{0.0245}\\ \cline{2-3}
& 2 & 0.0726 &\\ \cline{2-3}
& 3 & - &\\  \cline{2-3}
\hline
\multirow{3}{*}{8} & 1 & 0.0628 & \multirow{3}{*}{0.0693}& \multirow{3}{*}{0.0094}\\ \cline{2-3}
& 2 & 0.0797 &\\ \cline{2-3}
& 3 & 0.0653 &\\  \cline{2-3}
\hline
\multirow{3}{*}{12} & 1 & 0.0613 & \multirow{3}{*}{0.0618}& \multirow{3}{*}{0.0044} \\ \cline{2-3}
& 2 & 0.0664 &\\ \cline{2-3}
& 3 & 0.0577 &\\  \cline{2-3}
\hline
\multirow{3}{*}{24} & 1 & 0.0810 & \multirow{3}{*}{0.0671}& \multirow{3}{*}{0.0105}\\ \cline{2-3}
& 2 & 0.0641 &\\ \cline{2-3}
& 3 & 0.0617 &\\  \cline{2-3}
\hline
\multirow{3}{*}{48} & 1 & 0.0563 & \multirow{3}{*}{0.0395}& \multirow{3}{*}{0.0238}\\ \cline{2-3}
& 2 & 0.0227 &\\ \cline{2-3}
& 3 & - &\\  \cline{2-3}
\hline
\end{tabular}

\subsection{For $1500g$}
\begin{tabular}{|c|c|c|c|c|} \hline
Hour & Sample no. & Absorbance ($A$) & Average ($\bar{A}$) & S.D. ($\sigma_{A}$) \\
\hline \hline
\multirow{3}{*}{0} & 1 & 0.0698 & \multirow{3}{*}{0.0689} & \multirow{3}{*}{0.0013}\\ \cline{2-3}
& 2 & 0.0680 &\\ \cline{2-3}
& 3 & - &\\  \cline{2-3}
\hline
\multirow{3}{*}{4} & 1 & 0.0701 & \multirow{3}{*}{0.0738}& \multirow{3}{*}{0.0052}\\ \cline{2-3}
& 2 & 0.0775 &\\ \cline{2-3}
& 3 & - &\\  \cline{2-3}
\hline
\multirow{3}{*}{8} & 1 & 0.0669 & \multirow{3}{*}{0.0720}& \multirow{3}{*}{0.0072}\\ \cline{2-3}
& 2 & 0.0771 &\\ \cline{2-3}
& 3 & - &\\  \cline{2-3}
\hline
\multirow{3}{*}{12} & 1 & 0.0685 & \multirow{3}{*}{0.0665} & \multirow{3}{*}{0.0039}\\ \cline{2-3}
& 2 & 0.0691 &\\ \cline{2-3}
& 3 & 0.0620 &\\  \cline{2-3}
\hline
\multirow{3}{*}{24} & 1 & 0.0661 & \multirow{3}{*}{0.0730}& \multirow{3}{*}{0.0097}\\ \cline{2-3}
& 2 & 0.0798 &\\ \cline{2-3}
& 3 & - &\\  \cline{2-3}
\hline
\multirow{3}{*}{48} & 1 & 0.0545 & \multirow{3}{*}{0.0658}& \multirow{3}{*}{0.0160}\\ \cline{2-3}
& 2 & 0.0771 &\\ \cline{2-3}
& 3 & - &\\  \cline{2-3}
\hline
\end{tabular}

\subsection{For $2000g$}
\begin{tabular}{|c|c|c|c|c|} \hline
Hour & Sample no. & Absorbance ($A$) & Average ($\bar{A}$) & S.D. ($\sigma_{A}$) \\
\hline \hline
\multirow{3}{*}{0} & 1 & 0.0303 & \multirow{3}{*}{0.0378} & \multirow{3}{*}{0.0105}\\ \cline{2-3}
& 2 & - &\\ \cline{2-3}
& 3 & 0.0452 &\\  \cline{2-3}
\hline
\multirow{3}{*}{4} & 1 & 0.0469 & \multirow{3}{*}{0.0471}& \multirow{3}{*}{0.0002}\\ \cline{2-3}
& 2 & 0.0473 &\\ \cline{2-3}
& 3 & - &\\  \cline{2-3}
\hline
\multirow{3}{*}{8} & 1 & 0.0475 & \multirow{3}{*}{0.0480}& \multirow{3}{*}{0.0006}\\ \cline{2-3}
& 2 & 0.0484 &\\ \cline{2-3}
& 3 & - &\\  \cline{2-3}
\hline
\multirow{3}{*}{12} & 1 & 0.0597 & \multirow{3}{*}{0.0609} & \multirow{3}{*}{0.0017}\\ \cline{2-3}
& 2 & - &\\ \cline{2-3}
& 3 & 0.0621 &\\ 
\hline
\multirow{3}{*}{24} & 1 & - & \multirow{3}{*}{0.0622}& \multirow{3}{*}{0.0095}\\ \cline{2-3}
& 2 & 0.0555 &\\ \cline{2-3}
& 3 & 0.0689 &\\  \cline{2-3}
\hline
\multirow{3}{*}{48} & 1 & 0.0533 & \multirow{3}{*}{0.0537}& \multirow{3}{*}{0.0070}\\ \cline{2-3}
& 2 & 0.0608 &\\ \cline{2-3}
& 3 & 0.0469 &\\  \cline{2-3}
\hline
\end{tabular}

\subsection{For $2500g$}
\begin{tabular}{|c|c|c|c|c|} \hline
Hour & Sample no. & Absorbance ($A$) & Average ($\bar{A}$) & S.D. ($\sigma_{A}$) \\
\hline \hline
\multirow{3}{*}{0} & 1 & 0.0327 & \multirow{3}{*}{0.0259} & \multirow{3}{*}{0.0073}\\ \cline{2-3}
& 2 & 0.0267 &\\ \cline{2-3}
& 3 & 0.0182 &\\  \cline{2-3}
\hline
\multirow{3}{*}{4} & 1 & 0.0460 & \multirow{3}{*}{0.0465}& \multirow{3}{*}{0.0007}\\ \cline{2-3}
& 2 & 0.0470 &\\ \cline{2-3}
& 3 & - &\\  \cline{2-3}
\hline
\multirow{3}{*}{8} & 1 & 0.0561 & \multirow{3}{*}{0.0477}& \multirow{3}{*}{0.0094}\\ \cline{2-3}
& 2 & 0.0494 &\\ \cline{2-3}
& 3 & 0.0376 &\\  \cline{2-3}
\hline
\multirow{3}{*}{12} & 1 & 0.0793 & \multirow{3}{*}{0.0744} & \multirow{3}{*}{0.0055}\\ \cline{2-3}
& 2 & 0.0684 &\\ \cline{2-3}
& 3 & 0.0775 &\\  \cline{2-3}
\hline
\multirow{3}{*}{24} & 1 & 0.1015 & \multirow{3}{*}{0.0922}& \multirow{3}{*}{0.0132}\\ \cline{2-3}
& 2 & 0.0828 &\\ \cline{2-3}
& 3 & - &\\  \cline{2-3}
\hline
\multirow{3}{*}{48} & 1 & 0.0052 & \multirow{3}{*}{0.0060}& \multirow{3}{*}{0.0011}\\ \cline{2-3}
& 2 & 0.0068 &\\ \cline{2-3}
& 3 & - &\\  \cline{2-3}
\hline
\end{tabular}

\subsection{For $5000g$}
\begin{tabular}{|c|c|c|c|c|} \hline
Hour & Sample no. & Absorbance ($A$) & Average ($\bar{A}$) & S.D. ($\sigma_{A}$) \\
\hline \hline
\multirow{3}{*}{0} & 1 & 0.0439 & \multirow{3}{*}{0.0399} & \multirow{3}{*}{0.0044}\\ \cline{2-3}
& 2 & 0.0406 &\\ \cline{2-3}
& 3 & 0.0352 &\\  \cline{2-3}
\hline
\multirow{3}{*}{4} & 1 & 0.0833 & \multirow{3}{*}{0.0792}& \multirow{3}{*}{0.0129}\\ \cline{2-3}
& 2 & 0.0700 &\\ \cline{2-3}
& 3 & - &\\  \cline{2-3}
\hline
\multirow{3}{*}{8} & 1 & 0.0607 & \multirow{3}{*}{0.0617}& \multirow{3}{*}{0.0014}\\ \cline{2-3}
& 2 & 0.0627 &\\ \cline{2-3}
& 3 & - &\\  \cline{2-3}
\hline
\multirow{3}{*}{12} & 1 & 0.0443 & \multirow{3}{*}{0.0506} & \multirow{3}{*}{0.0088}\\ \cline{2-3}
& 2 & 0.0568 &\\ \cline{2-3}
& 3 & - &\\  \cline{2-3}
\hline
\multirow{3}{*}{24} & 1 & 0.1628 & \multirow{3}{*}{0.1521}& \multirow{3}{*}{0.0151}\\ \cline{2-3}
& 2 & 0.1414 &\\ \cline{2-3}
& 3 & - &\\  \cline{2-3}
\hline
\multirow{3}{*}{48} & 1 & 0.0428 & \multirow{3}{*}{0.0480}& \multirow{3}{*}{0.0045}\\ \cline{2-3}
& 2 & 0.0512 &\\ \cline{2-3}
& 3 & 0.0499 &\\  \cline{2-3}
\hline
\end{tabular}


\subsection{Estimated total protein content from standard graph}
From Fig. (12), we know that the equation of line is,
\begin{equation}
y= 956.6x-18.5
\end{equation} 
Thus,
\begin{equation}
\boxed{~\text{Total protein concentration} (\mu g/ml)= 956.6~( \text{absorbance})-18.5~} 
\end{equation}
Also, we know that if $x=Ay+B$, then $\sigma_{x}=A\sigma_{y}$. This helps us to convert error bars of BSA graph to total protein graph.\\
Thus,
\begin{equation}
\boxed{~\sigma_{\text{conc.}}=965.5~\sigma_{\text{absorbance}}~} 
\end{equation}
The estimated protein concentrations are tabulated below:\\
\begin{center}
\begin{tabular}{|c|c|c|c|c|}
\hline
\multirow{2}{*}{Hr.} & \multicolumn{4}{|c|}{Conc. of total protein ($\mu g/ml$)}\\ \cline{2-5}
& control & $500g$ & $1000g$ & $1500g$ \\
\hline \hline
0  & 0.0000 $\pm$ 0.1910 & 27.5681 $\pm$ 9.3747 &  0.1137 $\pm$ 5.1656 & 47.4097 $\pm$ 1.2436\\
\hline
4  & 6.4273 $\pm$ 4.7800 & 29.7683 $\pm$ 14.1577 & 34.3600 $\pm$ 23.4367 & 52.0971 $\pm$ 4.9743\\
\hline
8  & 0.0000 $\pm$ 3.9200 & 28.42906 $\pm$ 9.4703 & 47.7237 $\pm$ 8.9920 & 50.3752 $\pm$ 6.8875\\
\hline
12 & 29.6726 $\pm$ 12.8000 & 53.3007 $\pm$ 14.3490 & 40.5779 $\pm$ 4.2090  & 45.1139 $\pm$ 3.7307 \\
\hline
24 & 44.4999 $\pm$ 12.7000 & 45.9348 $\pm$ 3.3481 & 45.6479 $\pm$ 10.0443 & 51.3318 $\pm$ 9.2790\\
\hline
48 & 15.8019 $\pm$ 7.6500 & 27.1855 $\pm$ 16.8362 &  19.2457 $\pm$ 22.7671 & 44.4443 $\pm$ 15.3056\\ 
\hline
\end{tabular}
\end{center}
\vspace{0.2in}
\begin{center}
\begin{tabular}{|c|c|c|c|}
\hline
\multirow{2}{*}{Hr.} & \multicolumn{3}{|c|}{Conc. of total protein ($\mu g/ml$)}\\ \cline{2-4}
 & $2000g$ & $2500g$ & $5000g$\\
\hline \hline
0  &  17.6195 $\pm$ 10.0443 & 6.27594 $\pm$ 0.28698 & 19.66435 $\pm$ 4.20904  \\
\hline
4  &  26.5159 $\pm$ 0.1913 & 25.9819 $\pm$ 0.66962 & 57.2548 $\pm$ 12.34014\\
\hline
8  & 27.3768 $\pm$ 0.5740 & 27.12982 $\pm$ 8.99204  & 40.51605 $\pm$ 1.33924 \\
\hline
12 & 39.7169 $\pm$ 1.6262 & 52.67104 $\pm$ 5.2613  & 29.8989 $\pm$ 8.41808 \\
\hline
24 & 40.9605 $\pm$ 9.0877 & 69.69852 $\pm$ 12.62712 & 126.98365 $\pm$ 14.44466  \\
\hline
48 & 32.8294 $\pm$ 6.6962 & 0 $\pm$ 1.05226  & 27.412 $\pm$ 4.3047 \\ 
\hline
\end{tabular}
\end{center}
These readings have been plotted with and without error bars in figures(13) and (14) respectively.
\begin{figure}
\centering
 \includegraphics[width=20cm,height=20cm,angle=90]{protein_conc_noerr.ps}
 \caption{This is the concentrations of total protein (in $\mu g/ml$) for different post-imbibition hours and for different values of hypergravity. These are estimated from the standard graph for BSA. The error bars are not shown. The time at which compare readings is $24^{th}$ hour.}
\end{figure}
\\
\begin{figure}
\centering
 \includegraphics[width=20cm,height=20cm,angle=90]{protein_conc_err.ps}
 \caption{This is the concentrations of total protein (in $\mu g/ml$) for different post-imbibition hours and for different values of hypergravity. These are estimated from the standard graph for BSA. The error bars are shown. The time at which compare readings is $24^{th}$ hour.}
\end{figure}

\newpage
\section{Conclusions and future scope}
\begin{description}
\item[For amylase part]
Previous studies carried out in our laboratory showed hypergravity decreases percentage seed germination and growth in rice and wheat seedlings (Jagtap and Vidyasagar 2010). Amylases play very important role in seed germination. Thus,\textit{ a priori}, we would expect that the amylase content would decrease in hypergravity. Aim of this project was to study the effect of hypergravity on amylase activity. As seen from the amylase concentration versus time graph (Fig.10), amylase activity decreased in hypergravity samples $1000g$, $1500g$, $2500g$, and $5000g$ except in $500g$ and $2000g$ where increase in amylase activity was observed. This could be because of handling error. More observations need to be carried out to understand the effect of hypergravity conditions on amylase activity and to draw some definitive conclusions.\\
As we have seen the amylase production is dependent on the activation of the corresponding genes. The gene expression can change due to the mechanical stresses. It needs to be studied how the gene expressions are changed in hypergravity and microgravity. This is important because, like plants, humans too depend on amylase for the digestion of starch. And the same gene will code for amylase in humans. Thus, if we know how the gene expression is regulated in hypergravity and microgravity, we will have a good insight into what effects can be observed in humans.

\item[For protein part] 
As seen from the concentration of protein versus time graph (Fig. 13), highest protein content was observed at 24$^{th}$ hr in control as well as hypergravity exposed samples. Interestingly, protein content is higher in hypergravity samples than control sample. Increase in protein content in hypergravity samples could be because of activation some heat shock proteins (HSPs) due to the hypergravity stress. It has been hypothesized that particular proteins whose synthesis is induced by stress conditions are critical for the survival under these circumstances (Schoffl \textit{et al.} 1988; Scandalios 1990). For example, accumulation of Heat Shock Proteins (HSPs) is crucial in protecting from thermal killing during heat stress (Vierling 1991). Kozeko and Kordyum have observed changes in the levels of HSPs under hypergravity (Kozeko and Kordyum 2009) and clinorotation (Kozeko and Kordyum 2006) in pea.\\
There are variety of heat shock proteins observed in almost every organism. In future study, using Western blot analysis, the type of heat shock proteins can be determined. 
\end{description}
\newpage
\section{Acknowledgements}
When you work on a project for so many long hours that you lose track of time and months go by unnoticed, it is hard to quit working on it, since it has become a part of you.\\
 It was a privilege to work on a project which truly gave me a research experience (and introduced me to the woes of an experimentalist!). Agreed, I had my own share of frustrations because of the Murphy's law\footnote{Murphy's law: If something \textit{can} go wrong, it \textit{will}!}, but ultimately I am contended with the work I have done.\\
I am thankful to my guide, Prof. Vidyasagar, for accepting me as his project student. I appreciate that I was given freedom to carry out things my way. I feel lucky to have got Sagar Jagtap as a co-guide whose patience and sincerity towards everything that he does has astonished me and inculcated lessons which I would remember for years to come. I wish to thank Department of Physics for the excellent facilities which were made accessible to me.\\
My apologies to all my friends whose incessant requests to accompany them in some of the joyous moments were turned down in favor of lab-work, to my brother whom I could never join playing \textit{Prince of Persia} despite he so much wished to show off his skills at it, to my parents and family whose calls were never picked up during daytime.\\
If I have been able to do anything worthwhile in my life, it's because of you all. Thank you.
\newpage

%\addcontentsline {toc}{section} {Bibliography}
\begin{thebibliography}{99}
\bibitem{1} William G. Hopkins and Norman P. A. H"uner, ``\textit{Introduction to Plant Physiology},'' $3^{rd}$ Edition, John Wiley and Sons, 2004.
\bibitem{2} ``\textit{Biology},'' NCERT text-book for Classes XI and XII, 2006.
\bibitem{3} S. Bhaskaran, S. S. Jagtap, and P. B. Vidyasgar, ``\textit{Life and Gravity},'' Biophysical Reviews and Letters, Volume 4, Number 4, October 2009.
\bibitem{4} J. Derek Bewley and Michael Black, ``\textit{Seeds: Physiology of Development and germination},'' $2^{nd}$ Edition, Plenum Press, 1994.
\bibitem{5} Sheela Khanwani, ``\textit{Determination of $\beta$-amylase content in germinating phase of monocotyledons},''  Project report, Garden City College, Bangalore, 2004.
\bibitem{6} Jagtap S. S. and Vidyasagar P. B., ``\textit{Effects of high gravity ($g$) values on growth and chlorophyll content in wheat},'' Int. J.  Integ. Biol., 9(3):127-129, 2010. 
\bibitem{7} Kozeko, L., Kordyum, E., ``\textit{The stress protein level under clinorotation in context of the seedling developmental program and the stress response},'' Microgravity Sci. Technol. 18(3-4), 254-256 (2006) 
\bibitem{8} Kozeko, L., Kordyum, E., ``\textit{Effect of hypergravity on the level of heat shock proteins 70 and 90 in pea seedlings},'' Microgravity Sci. Technol. 21(1-2), 175-178, 2009. 
\bibitem{9} Scandalios, J. G., ``\textit{Response of plant antioxidant defence gene to environmental stress},'' Advances in Genetics, 28, 1-41, 1990.
\bibitem{10} Schoffl, F., Baumann, G., Raschke, E., ``\textit{The expression of heat shock genes-a model for the environmental stress response},'' in  \textit{Temporal and spatial Regulations of plant Genes}, Eds. B. Goldberg and D. P. S. Verna, Springer Verlag, Wein, 1988.
\bibitem{11} Vidyasagar P., Jagtap S., Nirhali A., Bhaskaran S., and Hase V., ``\textit{Effects of hypergravity on the chlorophyll content and growth of root and shoot during development in rice plants},''  in \textit{Photosynthesis Energy from the Sun: 14th International Congress on Photosynthesis}, J.F. Allen, E. Gantt, J.H. Golbeck, and B. Osmond (eds.),  Springer, 1597-1600, 2008. 
\bibitem{12} Vierling, E., ``\textit{The role of heat shock proteins in plants},'' Annu. Rev. Plant Physiol. Plant Mol. Biol. 42, 579-620, 1991.
\bibitem{13} www.wikipedia.org
\end{thebibliography}

\newpage
\begin{LARGE}\textbf{Appendices}\end{LARGE}
\addcontentsline{toc}{section}{Appendices}
\appendix
\setcounter{section}{0}
\section{Ways to alter gravity}
\subsection{Centrifuge}
In order to simulate the hypergravity conditions, the apparatus known as centrifuge can be used. Using it, the seeds were given the treatment of gravity for values ranging from $500g$ to $5000g$. And also, the enzyme assay was rotated at $10000g$ for sedimentation. Let's study this instrument in detail.\\
A centrifuge is a piece of equipment, generally driven by an electric motor, which puts an object in rotation around a fixed axis, applying a force perpendicular to the axis. The centrifuge works using the sedimentation principle, where the centripetal acceleration causes more dense substances to separate out along the radial direction (the bottom of the tube). By the same token, lighter objects will tend to move to the top.\\
The rotating unit, called the rotor, has fixed holes drilled at an angle (to the vertical). Centrifugation tubes are placed in these slots and the rotor is spun. As the centrifugal force is in the horizontal plane and the tubes are fixed at an angle, the particles have to travel only a little distance before they hit the wall and drop down to the bottom.\\ 
Protocols for centrifugation typically specify the amount of acceleration to be applied to the sample, rather than specifying a rotational speed such as revolutions per minute. This distinction is important because two rotors with different diameters running at the same rotational speed will subject samples to different accelerations. Since the motion is circular the acceleration can be calculated as the product of the radius and the square of the angular velocity. ($a=r\omega^{2}$)\\
  The acceleration is often quoted in multiples of $g$, the standard acceleration due to gravity at the Earth's surface. Traditionally named \textit{relative centrifugal force} (RCF), it is the measurement of the acceleration applied to a sample within a centrifuge and it is measured in units of $g$. It is given by,
\begin{equation}
RCF = r \frac{(2\pi N)^{2}}{g}
\end{equation} 
where\\
$g$ is earth's gravitational acceleration,\\
$r$ is the rotational radius,\\
$N$ is the rotational speed, measured in revolutions per unit of time.\\
When the rotational speed is given in revolutions per minute (RPM) and the rotational radius is expressed in centimetres (cm) the above relationship becomes
\begin{equation}
RCF = 1.118~r_{cm} {N_{RPM}}^{2}
\end{equation}
where\\
$r_{cm}$ is the rotational radius measured in centimetres (cm),\\
$N_{RPM}$ is rotational speed measured in revolutions per minute (RPM).

\subsection{Clinostat}
A clinostat is a device which uses rotation to negate the effects of gravitational pull (and, thus, simulate the microgravity environment in lab) on plant growth (gravitropism) and development (gravimorphism). It has also been used to study the effects of microgravity on cell cultures and animal embryos.\\
A single axis (or horizontal) clinostat consists of a disc attached to a motor. The disc is held vertically and the motor rotates it slowly at rates in the order of one revolution per minute. A plant (or corresponding sample to b studies) is attached to the disc so that it is held horizontally. The slow rotation means that the plant experiences a gravitational pull that is averaged over 360 degrees, thus approximating a weightless environment.  If the clinostat is at an angle from horizontal, a net gravity vector is perceived, the magnitude of which depends on the angle. This can be used to simulate lunar gravity ($ 1/6 g$) which requires an angle from the horizontal of about $10^{\circ}$, \textit{i.e.} $sin^{-1}(1/6)$.\\
A plant only reacts to gravity if the gravistimulation is maintained for longer than a critical amount of time, called the minimal presentation time (MPT). For many plant organs the MPT lies somewhere between 10 and 200 seconds, and therefore a clinostat should rotate on a comparable timescale in order to avoid a gravitropic response. However, presentation time is cumulative, and if a clinostat's rotation is repeatedly stopped at a single position, even for periods as short as 0.5 s, a gravitropic response can result. The presentation time for animals is one or two orders of magnitude faster than this, thus precluding the use of the slow rotation clinostat for most animal studies. However the fast rotation clinostat can be, and is, used for the study of animal cell cultures and embryos.

\newpage
\section{Starch hydrolysis}
\subsection{Polysaccharides and glycosidic bond}
Monosaccharide is a basic carbohydrate unit, \textit{e.g.} glucose ($C_{6}H_{12}O_{6}$). A disaccharide is formed by the condensation reaction of two monosaccharides in which a molecule of water is formed (from a hydroxyl, $OH$, from one monosaccharide and a hydrogen, $H$, from the other) and two units are connected by oxygen. An example of disaccharide is maltose ($C_{12}H_{22}O_{5}$), which results from condensation of two glucose units.\\
Starch is a polysaccharide, \textit{i.e.} it is made up of long chain of glucose\footnote{Glycogen is also a polymer of glucose; it differs from starch in the way in which the glucose molecules are joined together.}. It is represented by the chemical formula $(C_{6}H_{10}O_{5})_{n}$. In cells, sugars are always stored in the form of starch, which has a high molecular weight and insoluble in water. Once formed within cell, it is impossible for starch to move out of the cell. It degrades back into glucose whenever cell requires energy. This is done by hydrolyzing the starch in the presence of a suitable enzyme\footnote{This class of enzymes which catalyzes cleavage of bonds between carbon and some other atom are called as hydrolytic enzymes.}.\\
A glycosidic bond can join two monosaccharide molecules to form a disaccharide, as, for instance, in the linkage of
glucose and fructose to create sucrose. More complicated polysaccharides such as starch (an important nutrient),
glycogen, cellulose (cell walls of plants), or chitin (found in fungi) consist of numerous monosaccharide units joined
by glycosidic bonds. While the cyclic structures of monosaccharide units are fairly rigid, the glycosidic bonds confer flexibility to polysaccharide molecules.\\
Glycosidic bonds join monosaccharides to form polysaccharides, just like peptide bonds join amino acids to form
proteins. The conformation of the torsional angles about the glycosidic bond are the most flexible point of a
polysaccharide. The orientations of these torsions typically fall within an expected range of values.

\subsection{Amylase}
Amylase is a hydrolytic enzyme that breaks starch down into sugar. Amylase is present in human saliva, where it begins the chemical process of digestion. Foods that contain much starch but little sugar, such as rice and potato, taste slightly sweet as they are chewed because amylase turns some of their starch into sugar in the mouth. The pancreas also
makes amylase ($\alpha$-amylase) to hydrolyse dietary starch into disaccharides and trisaccharides which are converted by other enzymes to glucose to supply the body with energy. Plants and some bacteria also produce amylase. 

\begin{description}
\item[$\alpha$-amylases] are calcium metalloenzymes, completely unable to function in the absence of calcium. By acting at random locations along the starch chain, $\alpha$-amylase breaks down long-chain carbohydrates, ultimately yielding maltotriose and maltose from amylose, or maltose, glucose and limit dextrin from amylopectin. Because it can act anywhere on the substrate, $\alpha$-amylase tends to be faster-acting than $\beta$-amylase. In animals, it is a major digestive enzyme and its optimum pH is 6.7-7.0.\\
In human physiology, both the salivary and pancreatic amylases are $\alpha$-amylases.

\item[$\beta$-amylases] work from the non-reducing end and catalyze the hydrolysis of the second $\alpha-1,4$ glycosidic bond, cleaving off two glucose units (maltose) at a time. During the ripening of fruit, $\beta$-amylase breaks starch into maltose, resulting in the sweet flavor of ripe fruit.\\
Both $\alpha$-amylase and $\beta$-amylase are present in seeds; $\beta$-amylase is present in an inactive form prior to germination, whereas $\alpha$-amylase and proteases appear once germination has begun. Cereal grain amylase is key to the production of malt. Many microbes also produce amylase to degrade extracellular starches. Animal tissues do not contain $\beta$-amylase, although it may be present in microrganisms contained within the digestive tract.
\end{description}

\begin{tabular}{|c|c|}
\hline
Enzyme & \textit{Modus operandi} \\
\hline \hline
$\alpha$-amylase & hydrolyzes glycosidic bonds of the substrate \\
\hline
$\beta$-amylase & hydrolyzes \textit{maltose} units from the non-reducing terminal end \\
\hline
glucoamylase & hydrolyzes \textit{glucose} units from the non-reducing terminal end \\
\hline
\end{tabular}

\subsection{Pathways of starch catabolism}
The hydrolytic catabolic pathway is discussed in this section. There is also a phosphorolytic pathway, but we will not discuss it here.\\
The amylose and amylopectin in the native starch grain are first hydrolyzed by $\alpha-$amylase, which breaks the $\alpha(1\rightarrow4)$ glycosidic links between the glucose residues randomly throughout the chains and the products are hydrolyzed till the end products are glucose and maltose:
\begin{eqnarray}
 \addtolength{\fboxsep}{5pt}
\boxed{
\begin{gathered}
\text{Amylose} \xrightarrow{\alpha-amylase} \text{Glucose}+\text{Maltose}\\
\text{Amylopectin} \xrightarrow{\alpha-amylase} \text{Glucose}+\text{Maltose}+\text{limit dextrin}\\
\text{limit dextrin} \xrightarrow{\alpha-amylase} \text{Glucose}+\text{Maltose}
\end{gathered}
}
\end{eqnarray}

$\beta-$amylase can not hydrolyze native starch grains; rather it cleaves away successive maltose units from the non-reducing end of large oligomers released by prior $\alpha-$amylolytic attack:
\begin{eqnarray}
\addtolength{\fboxsep}{5pt}
\boxed{
\begin{gathered}
\text{Amylose} \xrightarrow{\beta-amylase} \text{Maltose}\\
\text{Amylopectin} \xrightarrow{\beta-amylase} \text{Maltose}+\text{limit dextrin}
\end{gathered}
}
\end{eqnarray}

Thus, $\alpha-$amylase gives maltose and glucose, while $\beta-$amylase gives only maltose as the end products of starch hydrolysis. Hence, when we measure maltose concentrations in our samples, we are tracing down the cumulative activity of both amylase.

\newpage
\section{Spectrophotometer}
A spectrophotometer is a photometer (a device for measuring light intensity) that can measure intensity as a function of the light source wavelength. The most common application of spectrophotometers is the measurement of light absorption. Let's first understand the Beer-Lambert's law of absorption.

\subsection{Beer-Lambert's law}
This law basically relates the absorption of light to the properties of the material through which light is travelling. Historically, the Lambert law states that absorption is proportional to the light path length, whereas the Beer law states that absorption is proportional to the concentration of absorbing species in the material.\\
In its modern incarnation, Beer-Lambert law states that there is a logarithmic dependence between the transmission of light through a substance and the product of the absorption coefficient of the substance ($\alpha$), and the distance that light travels through the material ($l$). Thus, absorbance of the given substance is given by,
\begin{equation}
\boxed{A = \alpha l = \varepsilon c=-log_{10}(\frac{I}{I_{0}})~~,}
\end{equation}   
where\\ 
$I_{0}$ is the intensity of the incident beam,\\
 $I$ is the intensity of the out-coming light from sample,\\
 $\varepsilon$ is the absorptivity of the absorber,\\
 $c$ is the concentration of the absorbing species in the material.\\
 This implies that the absorbance becomes linear with the concentration for a given solution. Thus, more the concentration, more is the absorbance. This is the basis of the working of spectrophotometer. 
 
 \subsection{Working of spectrophotometer}
 There are two major classes of devices: single beam and double beam. A double beam spectrophotometer compares the light intensity between two light paths, one path containing a reference sample and the other the test sample. A single beam
spectrophotometer measures the relative light intensity of the beam before and after a test sample is inserted. Although comparison measurements from double beam instruments are easier and more stable, single beam instruments can have a larger dynamic range and are optically simpler and more compact.\\
The spectrophotometer quantitatively compares the fraction of light that passes through a reference solution and a
test solution. Light from the source lamp is passed through a monochromator, which diffracts the light into a
\textit{rainbow} of wavelengths and outputs narrow bandwidths of this diffracted spectrum. Discrete frequencies are
transmitted through the test sample. Then the intensity of the transmitted light is measured with a photodiode or
other light sensor, and the transmittance value for this wavelength is then compared with the transmission through a
reference sample.\\
In short, the sequence of events in a spectrophotometer is as follows:\\
1. The light source shines through a monochromator.\\
2. An output wavelength is selected and beamed at the sample.\\
3. A fraction of the monochromatic light is transmitted through the sample and to the photodetector.\\
In the current experiment, we resorted to double beam spectrophotometer and measured absorbances of various samples as a function of varying concentrations. Specifications of the instrument we use:\\
\begin{verbatim}
Instrument Name: JASCO UV-VIS-NIR Spectrophotometer
Model Name: V-670
Wavelengths used (in nm): 540.0, 590.0 
\end{verbatim} 

\newpage
\section{List of chemicals required}
Below is an exhaustive list of all the chemicals that were required during the course of the experiment:
\\
\begin{tabular}{|c|c|}
\hline
Name & Chemical Formula \\
\hline \hline
Starch & $(C_{6}H_{10}O_{5})_{n}$ \\
\hline
Sodium chloride & $NaCl$ \\
\hline
Sodium phosphate dibasic & $Na_{2}HPO_{4}$\\
\hline
Sodium dihydrogen orthophosphate & $NaH_{2}PO_{4}$ \\
\hline
Sodium potassium tartarate 	& $COOK.CH(OH).COONa$ \\
\hline
Sodium hydroxide &	$NaOH$ \\
\hline
DNSA (3,5-dinitrosalicylic acid)  &	$C_{7}H_{4}N_{2}O_{7}$ \\
\hline
Maltose & $C_{12}H_{22}O_{11}$ \\
\hline
CBB (coomasine brilliant blue) G250 & $C_{47}H_{60}N_{3}O_{7}S_{2}$ \\
\hline
Ethanol &	$C_{2}H_{5}OH$ \\
\hline
Phosphoric acid & $H_{3}PO_{4}$ \\
\hline
TTC (2,3,5-triphenyltetrazolium chloride) &	$C_{19}H_{15}CIN_{4}$ \\
\hline
Distilled water &	$H_{2}O$ \\
\hline
BSA (bovine serum albumin) & - \\
\hline
Agar-agar powder & - \\
\hline
Fungicide & - \\
\hline
\end{tabular}


\end{normalsize}
\end{document}





